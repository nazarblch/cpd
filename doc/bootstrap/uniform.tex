Quadratic approximation of $\alpha(\theta_0, \theta)$ is correct with high probability for each window position $t$, but it takes place uniformly for all the windows with higher probability than independent events because of correlations. Corresponding probability of the uniform approximation could be obtained by grouping window positions into high correlation regions. 
After split the whole data range of size $T$ into such regions of size $r_t$ the uniform approximations are to be true for all the regions with probability $1 - e^{-x_t}$, $x_t = x - \log(T/r_t)$. 
The first grouping approach uses bootstrap related equations.  
\[
L(\theta, t) = \sum_{i=1}^{n} l_i(\theta) u_i(t),
\quad
1 \leq t \leq n = h + 2r_t,
\]
where $u_i(t)$ denotes kernel function that selects elements around coordinate $t$ with distance less than $h/2$: 
\[
u_i(t) = \Ind\{|t - i| \leq h/2\}, 
\quad \E_t u_i(t) = \frac{h}{n}.
\]
Weights $u_i(t)$ have probabilistic nature since they are dependently generated by sampling position $t \in [1,n]$ uniformly.  The expectation comes from all window positions of circle closed data ($Y_{n+i} = Y_i$). 
\[
\alpha(t, \theta, \theta_0) = L(\theta, t) -  L(\theta_0, t)  - (\theta - \theta_0)  \nabla  L(\theta_0, t) + \frac{1}{2} \Vert D_h (\theta - \theta_0) \Vert^2 
\]
Precise quadratic approximation requires small value of $ \nabla \alpha(t, \theta, \theta_0) $ for all  From (\ref{cond_A}) and weights mean follows
\[
\Vert \E_t D_{h}^{-1} \nabla \alpha(t) \Vert \leq \sqrt{\frac{h}{n}} \cdot 2 \diamondsuit(r(\theta^*), x, n) \sim \sqrt{\frac{h}{n}} \sqrt{\frac{p}{n}}.
\]
Defining variable similar to (\ref{S}) one obtains its mean and deviation bounds  in separate steps.
\[
S(t) = D_h^{-1} \{\nabla \alpha(t) -  \E_t \nabla \alpha(t) \},
\quad 
\Vert S(t) - \E_{Y} S(t) \Vert \leq \sqrt{1 - \frac{h}{n}} \cdot 12 \nu_0 z(x,p) \omega(h) \max_i r(\theta^*_i).
\]
Here $r(\theta^*_i)$ denotes max distance from $\argmax \E l_i (\theta)$ to $(\theta, \theta_0)$. From asymptotic assumptions for $\omega(h) \sim 1 /\sqrt{h}$ and $z(x) \sim \sqrt{p}$	 one get 
\[
\Vert S(t) - \E_Y S(t) \Vert \sim \sqrt{\frac{p}{h}} \overline{r}.
\]
The mean part requires additional condition (\ref{Li}) for divergence of the likelihood components     
\[	
\Vert \E_Y S(t) \Vert \leq  \max_i \frac{C(r_i)}{h} r_i  \sum_{j=1}^{n} | u_j - \E_t u_j |  = 2 \max_i C(r_i) r_i \left( 1 - \frac{h}{n}  \right),
\]
where 
\[
 \sum_{j=1}^{n} | u_j - \E_t u_j |  = (n - h) \left | 0 - \frac{h}{n} \right | +  h \left | 1 - \frac{h}{n} \right | = 2 h \left(1- \frac{h}{n} \right).
\]
Finally, the uniform estimation error for region $n$ is
\[
 12 \nu_0 z(x,p) w(h) \max_i r(\theta^*_i) + 2 \max_i C(r_i) r_i \left( 1 - \frac{h}{n}  \right).
\]
From which with $C(r_i) \sim 	r_i$ one get  $ 1 / \sqrt{h} \sim  (1- h/n) $, consequently  $n - h \sim \sqrt{h}$.

The second grouping approach treats window position $t$ as a part of parameter $\theta$. Define variable $y$ as difference of stochastic likelihood part gradients: 
\[
y(\theta, t) = D_h^{-1} \left( \nabla \zeta(\theta,t) - \nabla \zeta(\theta_0,t_0) \right).
\]
Define new parameters $u$, $\tau$ aiming for restriction $\Vert u \Vert \leq r_0$, $\Vert \tau \Vert \leq r_0$:
\[
u = D_h (\theta - \theta_0), 
\quad
\tau = \frac{r_0}{r_t} (t - t_0),
\quad
|t- t_0| \leq r_t.
\]
Estimation for $r_t$ comes from equating parts of $y$ gradients from each parameter.
\[
\nabla_u y = D_h^{-1}  \nabla^2 \zeta(\theta,t) D_h^{-1},
\quad
\nabla_{\tau} y = \frac{r_t}{r_0} D_h^{-1} \nabla_t \nabla \zeta(\theta,t)
\]
\[
\Vert D_h^{-1} \nabla_t \nabla \zeta(\theta,t) \Vert \leq 2 \Vert D_h^{-1}  \nabla  l_i(\theta_i^*,t)  \Vert + \Vert D_h^{-1}  \nabla^2  \zeta_i(\cdot ,t) D_h^{-1} \Vert r_0 \sim \sqrt{\frac{p}{h}}.
\]
\[
\Vert D_h^{-1}  \nabla^2 \zeta(\theta,t) D_h^{-1} \Vert 	\sim  \omega(h) z(x) \sim \sqrt{\frac{p}{h}}.
\]
Finally, this approach results in $n - h = r_t \sim r_0$.


  