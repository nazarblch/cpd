Provide simple non-strict explanation of what kind of distribution the main statistic  $\dLh$ is supposed to have. Review  $\dLh$ as
\[
\dLh = L(\widehat{\theta}) -  L(\widehat{\theta}_{H_0}), 
\quad L(\theta) = L_1(\theta_1) + L_2(\theta_2), 
\quad L_1 = L(Y[1:h]), \; L_2 = L(Y[h:2h]), 
\]
where $\widehat{\theta}_{H_0}$ is argmax of $L$ under condition $H_0: \theta_1 =\theta_2$. Then under quadratic approximation assumption $\dLh$ could be presented in Tailor equation with point $\widehat{\theta}$:
\[
\dLh \approx \frac{1}{2} \Vert D(\widehat{\theta} - \widehat{\theta}_{H_0})  \Vert^2.
\]   
If $\widehat{\theta}$ and $\widehat{\theta}_{H_0}$ tend to be Normal and $H_0$ is true then their difference are close to a centered Normal variable. If $H_0$ is false -- the Normal variable will have mean that  is equal to $\theta^* - \theta_{H_0}^*$. 

The next two theorems includes more formal properties of $\dLh$ statistic.  

\begin{theorem}
\label{dl_theorem}
Assume condition (\ref{cond_L_star}) and quadratic Laplace approximation (\ref{cond_A}) of $L_1$ and $L_2$   are fulfilled with probability $1 - 2 e^{-x}$, additionally with probability $1 - 2 e^{-x}$
\[
\Vert \xi_i \Vert \leq z(x), 
\quad z^2(x) = \max_i p_{B_i} + 6 \lambda_{B_i} x,
\] 
\[\tag{B}
B_i = D_i^{-1} \Var (\nabla L_i(\theta^*))D_i^{-1},
\quad p_B = \tr(B), 
\quad \lambda_B  = \lambda_{\max} (B).
\]
Then in the local region with probability $1 - 8 e^{-x}$ 
\[
2 \dLh = \Vert  \dxi + \dtheta \Vert^2  + O(\{r + z(x)\} \diamondsuit (r, x)),
\]
where
\[
\dxi  = \Sigma (D_2^{-1} \xi_2 - D_1^{-1} \xi_1),
\quad
\dtheta  = \Sigma (\theta_2^* - \theta_1^*).
\]
\end{theorem}    

\begin{remark} 
\label{dxi_limit}
In increasing sample size $n \to \infty$ the stochastic component tends to Normal distribution: 
\[
\dxi \to \mathcal{N}(0, B_1 + B_2).
\]
\end{remark}

\begin{remark}
For the condition $\widehat{\theta} \in \Theta_1(r) \cap \Theta_2(r)$ the restriction of the parameter variability $\theta^*$ is required
\begin{equation}\label{cond_L_star}
\tag{L*}
\Vert D(\theta_1^* - \theta_2^*) \Vert \leq r.
\end{equation}
\end{remark}


\begin{theorem}
\label{dl_sq_theorem}
Assume condition (\ref{cond_L_star}) and quadratic Laplace approximation (\ref{cond_A}) with probability $1 - 2 e^{-x}$ are fulfilled. Then  with probability $1 - 4 e^{-x}$ in the local region  $\Theta_1(r) \cap \Theta_2(r)$ took place
\[
\left| 
\sqrt{ 2\dLh} - 
\Vert \dxi + \dtheta \Vert 
\right| \leq 
10  \diamondsuit (r, x).
\]
where $\dxi$ and $\dtheta$ are defined in theorem \ref{dl_theorem}.
\end{theorem}

