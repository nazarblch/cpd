This section explicates main restrictions of the likelihood function. They are essential for the proposed algorithm of change point detection. Further assume that log-likelihood function $L(\theta) = L(Y,\theta)$,  $Y = (Y_1,\ldots,Y_n)$, has rather precise approximation by its quadratic part in local region $\localr$  of $\theta^*$, $\localr \subseteq \R^p$, where
\[
\theta^* = \argmax_{\theta} \E L(\theta),
\quad
\widehat{\theta} = \argmax_{\theta} L(\theta)
\]
and $\localr = \{\Vert D (\theta - \theta^*) \Vert < r \}$. \cite{wilks2013} provides required conditions for justified quadratic approximation and parameter concentration in the local region.
Approximation error involves the next variables for its estimation:
\[
  \alpha(\theta, \theta_0) = L(\theta) - L(\theta_0)   - (\theta - \theta_0)^T \gradL( \theta_0) +  \frac{1}{2} \Vert D (\theta - \theta_0) \Vert^2, 
\]
\[
\chi(\theta, \theta_0) = D^{-1} \nabla \alpha(\theta, \theta_0) 
= D^{-1} (\gradL(\theta) - \gradL( \theta_0) ) +  D (\theta - \theta_0). 
\]
Let  in region $\Theta_0(r)$ with probability $1 - e^{-x}$:
\begin{equation}\label{cond_A}\tag{A}
\frac{| \alpha(\theta, \theta^*)  |}{\Vert D(\theta - \theta^*) \Vert} \leq \diamondsuit (r, x),  
\quad
  \Vert \chi(\theta, \theta^*) \Vert \leq  \diamondsuit (r, x),
\end{equation}
where $\diamondsuit (r, x) = (\delta (r) + 6 v_0 z_H(x) \omega ) r,$
\[\tag{D}
D^2(\theta) = - \nabla^2 \E L (\theta),
\quad
D = D(\theta^*),
\]
\begin{equation}\label{cond_dD}\tag{dD}
\Vert D^{-1} D^2(\theta) D^{-1} - I_p\Vert \leq \delta(r),
\end{equation}
\begin{equation}\label{cond_ED2}\tag{ED2}
\forall \lambda \leq g, \; \gamma_1 \gamma_2 \in \R^p: \quad
\log \E \exp \left\{
\frac{\lambda}{\omega} \frac{\gamma_1^T \nabla^2 \zeta(\theta) \gamma_2}{\Vert D \gamma_2 \Vert \Vert D \gamma_2 \Vert}
\right\} \leq 
\frac{\nu_0^2 \lambda^2}{2},
\end{equation}
\[
z_H(x) = \sqrt{H} + \sqrt{2x} + \frac{g^{-2} x + 1}{g} H, 
\quad H = 6p,
\]
\[
\omega = \omega(n) \sim \frac{1}{\sqrt{n}}, 
\quad
\delta(r) \sim \frac{r}{\sqrt{n}},
\quad
r \sim \sqrt{p}.
\]
Condition (\ref{cond_dD}) ensures quadratic approximation of $\E L(\theta)$ and (\ref{cond_ED2}) ensures linear approximation of centered likelihood $\zeta(\theta) = L(\theta) - \E L(\theta)$.  