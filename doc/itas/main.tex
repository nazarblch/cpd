\documentclass{style/llncs}
\usepackage{style/llncsdoc}

\usepackage[intlimits]{amsmath}
\usepackage{amsthm,amsfonts}
\usepackage{amssymb}
\usepackage{mathrsfs}
%\usepackage{graphicx}
\usepackage[final]{graphicx,epsfig} 
\usepackage{longtable}
\usepackage{indentfirst}
\usepackage[utf8]{inputenc}
\usepackage[]{algorithm2e}
\usepackage[english]{babel}
\usepackage[usenames]{color}
\usepackage{algorithm2e}

\usepackage{hyperref}

\renewcommand{\thesection}{\arabic{section}}
\renewcommand{\thesubsection}{\arabic{section}.\arabic{subsection}.}


\usepackage{natbib}
\bibliographystyle{unsrtnat}
 
\title{Bibliography management: \texttt{natbib} package}
\author{Share\LaTeX}
%\usepackage{cite}

\graphicspath{{images/}}
\newcommand{\imgh}[3]{\begin{figure}[!h]\center{\includegraphics[width=#1]{#2}}\caption{#3}\label{Fig:#2}\end{figure}}

\usepackage{verbatim}

%\renewcommand{\)}{\,\end{math}}
\renewcommand{\(}{$\,}
\renewcommand{\)}{\,$}

\def\gpsii{g(\psi_i^{T} \theta)}
\def\gpsiidev{g''(\psi_i^{T} \theta)}
\def\rb0{r_0^{\flat}}

\def\nquad{\hspace{-1cm}}
\def\eqdef{\stackrel{\operatorname{def}}{=}}
\def\tod{\stackrel{d}{\longrightarrow}}
\def\tow{\stackrel{w}{\longrightarrow}}
\def\toP{\stackrel{\P}{\longrightarrow}}

\def\tS{\tilde{S}}
\def\Sv{\bf{S}}
\def\xiv{\bf{\xi}}
\def\tXv{\tilde{\Xv}}
\def\tX{\tilde{X}}
\def\GFC{\cc{G}_{3}}
\def\bdelta{\delta^{\sbt}}
\def\bqq{\qq^{\sbt}}
\def\tdelta{\err}%{\tilde{\delta}}
\def\tqq{\tilde{\qq}}
\def\eps{\varepsilon}
\def\fs{f}
\def\vp{\mathrm{v}}
\def\CONST{C}
\def\teps{\tilde{\varepsilon}}
\def\ND{\mathcal{N}}


\newcommand{\cc}[1]{\mathscr{#1}}
%\newcommand{\cc}[1]{\mathcal{#1}}
\newcommand{\bb}[1]{\boldsymbol{#1}}

%\renewcommand{\bar}[1]{\overset{\!\_\!\_\!\_}{#1}}
\renewcommand{\bar}[1]{\overline{#1}}
\renewcommand{\hat}[1]{\widehat{#1}}
\renewcommand{\tilde}[1]{\widetilde{#1}}


\renewcommand{\Gamma}{\varGamma}
\renewcommand{\Pi}{\varPi}
\renewcommand{\Sigma}{\varSigma}
\renewcommand{\Delta}{\varDelta}
\renewcommand{\Lambda}{\varLambda}
\renewcommand{\Psi}{\varPsi}
\renewcommand{\Phi}{\varPhi}
\renewcommand{\Theta}{\varTheta}
\renewcommand{\Omega}{\varOmega}
\renewcommand{\Xi}{\varXi}
\renewcommand{\Upsilon}{\varUpsilon}
%
\def\nn{\nonumber \\}

\def\suml{\sum\limits}
\def\supl{\sup\limits}
\def\maxl{\max\limits}
\def\infl{\inf\limits}
\def\intl{\int\limits}
\def\liml{\lim\limits}
\def\Cov{\operatorname{Cov}}
\def\Var{\operatorname{Var}}
\def\arginf{\operatornamewithlimits{arginf}}
\def\argsup{\operatornamewithlimits{argsup}}
\def\argmax{\operatornamewithlimits{argmax}}
\def\argmin{\operatornamewithlimits{argmin}}
\def\val{\operatorname{val}}

%\def\E{\boldsymbol{E}}
%\def\P{\boldsymbol{P}}
\def\D{\boldsymbol{D}}
\def\dd{\operatorname{d}}
\def\tr{\operatorname{tr}}
\def\I{I\!\!I}
\def\R{I\!\!R}
\def\E{I\!\!E}
\def\P{I\!\!P}
\def\X{\mathfrak{X}}
\def\kappa{\varkappa}
%\def\R{\mathbb{R}}
\def\Const{\mathrm{Const.} \,}
\def\cdt{\boldsymbol{\cdot}}
\def\tm{\!\times\!}
\def\T{\top}
\def\diag{\operatorname{diag}}
\def\bldiag{\operatorname{blockDiag}}
\def\diam{\operatorname{diam}}
\def\rank{\operatorname{rank}}
\def\loc{\operatorname{loc}}

\def\av{\bb{a}}
\def\bv{\bb{b}}
\def\cv{\bb{c}}
\def\dv{\bb{d}}
\def\ev{\bb{e}}
\def\fv{\bb{f}}
\def\gv{\bb{g}}
\def\hv{\bb{h}}
\def\iv{\bb{i}}
\def\jv{\bb{j}}
\def\kv{\bb{k}}
\def\lv{\bb{l}}
\def\mv{\bb{m}}
\def\nv{\bb{n}}
\def\ov{\bb{o}}
\def\pv{\bb{p}}
\def\qv{\bb{q}}
\def\rv{\bb{r}}
\def\sv{\bb{s}}
\def\tv{\bb{t}}
\def\uv{\bb{u}}
\def\vv{\bb{v}}
\def\wv{\bb{w}}
\def\xv{\bb{x}}
\def\yv{\bb{y}}
\def\zv{\bb{z}}

\def\Cv{\bb{C}}
\def\Gv{\bb{G}}
\def\Mv{\bb{M}}
\def\Sv{\bb{S}}
\def\Uv{\bb{U}}
\def\Xv{\bb{X}}
\def\Yv{\bb{Y}}
\def\Zv{\bb{Z}}


\def\alphav{\bb{\alpha}}
\def\betav{\bb{\beta}}
\def\gammav{\bb{\gamma}}
\def\deltav{\bb{\delta}}
\def\varepsilonv{\bb{\varepsilon}}
\def\epsv{\bb{\varepsilon}}
\def\etav{\bb{\eta}}
\def\gammav{\bb{\gamma}}
\def\phiv{\bb{\phi}}
\def\psiv{\bb{\psi}}
\def\tauv{\bb{\tau}}
\def\upsilonv{\bb{\upsilon}}
\def\xiv{\bb{\xi}}
\def\zetav{\bb{\zeta}}

\def\Psiv{\bb{\Psi}}
\def\CONST{\mathtt{C}}

\def\dL{T}
\def\dLh{T_h}
\def\dLhb{T_h^{\flat}}
\def\dLb12{T_h^{\flat}(\theta_1^{\flat}, \theta_2^{\flat})}
\def\dLht{T_h(t)}
\def\dLhconv{\mathbb{T}_h(\tau)}
\def\dxi{\xi_{12}}
\def\dxib{\xi^{\flat}_{12}}
\def\dtheta{\theta_{12}^*}
\def\dthetahat{\widehat{\theta}_{12}}
\def\localr{\Theta_0(r)}
\def\gradL{\nabla L}

\def\thetab{\theta^{\flat}}
\def\opttheta{\widehat{\theta}}
\def\opteta{\widehat{\eta}}

\def\txi{\tilde{\xi}}
\def\txib{\tilde{\xi}^{\flat}}

\def\rombb{\diamondsuit^{\flat}(r,x)}
\def\romb{\diamondsuit(r,x)}

\def\mle{\widehat{\theta}}
\def\reftheta{\theta^*}
\def\diagPsi{D(\Psi)}

\def\alphab{\alpha^{\flat}}
\def\chb{\chi^{\flat}}
\def\alphab12{\alpha^{\flat}(\theta, \theta_0)}
\def\chib12{\chi^{\flat}(\theta, \theta_0)}
\def\Lbf{L^{\flat}(\theta)}
\def\Lb0{L^{\flat}(\theta_0)}
\def\Lbopt{L^{\flat}(\widehat{\theta})}
\def\Lfopt{L(\widehat{\theta})}
\def\Lf{L(\theta)}
\def\L0{L(\theta_0)}

\def\Eb{\E_{\flat}}
\def\Varb{\Var_{\flat}}

\def\itemv{\vfill\item}
\newenvironment{myslide}[1]
    {\begin{frame}\frametitle{#1}\vfill}
    {\vfill\end{frame}}

\def\vsp{\vspace{0.05\textheight} \vfill}
\def\summarysign{\resizebox{0.08\textwidth}{0.08\textheight}{\includegraphics{summary}}\,}
\def\nix{}
\def\wpu{$\bullet$}

\def\btri{\vfill{\( \blacktriangleright \) }}
\def\btrir{\vfill{\( \blacktriangleright \) }}

\newcommand{\mygraphics}[3]{\begin{center}
    \resizebox{#1\textwidth}{#2\textheight}{\includegraphics{#3}}
    \end{center}
}

\newcommand{\mybox}[3]{\begin{center}
    \resizebox{#1\textwidth}{#2\textheight}{#3}
    \end{center}
}
\newcommand{\tobedone}[1]{\par\textbf{\color{red}To be done:} {\color{magenta}#1}}
\def\tobechecked{\qquad{\color{red} \text{\( (^{*}) \) please check}}}

%\definecolor{myhcolor}{rgb}{0.2,0,0.8}
%\definecolor{myhcolor}{named}{red}
\newenvironment{eqnh}
{
    %\color{myhcolor}} {}
    \setbeamercolor{postit}{fg=black,bg=hellgelb} %{fg=myhcolor,bg=white}
    \begin{beamercolorbox}[center,wd=\textwidth]{postit} %rounded=true,shadow=true,
    \begin{eqnarray*}}
    {\end{eqnarray*}\end{beamercolorbox}
}


\newcommand{\normp}[1]{
\left\Vert #1 \right\Vert
}

\newcommand{\normop}[1]{
\left\Vert #1 \right\Vert_{\oper}
}

\newcommand{\pr}[1]{
\P\left( #1 \right)
}


\def\Bernoulli{\mathrm{Bernoulli}}
\def\Vola{\mathrm{Vola}}
\def\Poisson{\mathrm{Poisson}}
\def\ag{\mathrm{ag}}
\def\glob{\operatorname{glob}}
\def\blk{\operatorname{block}}
\def\lin{\operatorname{lin}}
\def\cond{\, \big| \,}

\def\rdl{\epsilon}
\def\rd{\bb{\rdl}}
\def\rddelta{\delta}
\def\rdomega{\varrho}
\def\rddeltab{\rddelta^{*}}
\def\rhorb{\rhor^{*}}




\def\wv{\bb{w}}
\def\varthetav{\bb{\vartheta}}
\def\Lr{\breve{L}}
\def\zetavr{\breve{\zetav}}
\def\etavr{\breve{\etav}}
\def\xivr{\breve{\xiv}}


\def\rdb{\rd}
%\def\rdm{\bb{\sigma}}
\def\rdm{\underline{\rdb}}

\def\taub{\tau_{\rdb}}
\def\taum{\tau_{\rdm}}
\def\kappab{\kappa_{\rd}}
\def\deltab{\delta_{\rd}}

\def\taubGP{\tau_{\rdb,\GP}}
\def\taumGP{\tau_{\rdm,\GP}}
\def\kappabGP{\kappa_{\rd,\GP}}
\def\deltabGP{\delta_{\rd,\GP}}
\def\nubm{\nu_{\rd}}
\def\uub{u_{\rd}}
\def\uubGP{u_{\rd,\GP}}
\def\nubmGP{\nu_{\rd, G}}


\def\rG{\rd,\GP}

\def\LinSp{\mathrm{L}}
\def\Id{I\!\!\!I}
\def\Ind{\operatorname{1}\hspace{-4.3pt}\operatorname{I}}

\def\BG{\mathcal{R}}
\def\bg{r}
\def\fmup{\phi}
\def\rg{r}
\def\uc{u_{c}}
\def\muc{\mu_{c}}
\def\mud{\mu_{0}}
\def\xxd{\xx_{0}}
\def\yyd{\yy_{0}}
\def\gmd{\gm_{0}}

\def\ms{m^{*}}
\def\Inv{A}
\def\InvT{\Inv^{\T}}
\def\Invt{\tilde{\Inv}}


\def\ssize{N}
\def\nsize{{n}}

%\def\rhor{\mathfrak{b}}
\def\rhor{\omega}


\def\LT{L}
\def\LGP{\LT_{\GP}}
%\def\La{\breve{L}}
\def\La{\mathbb{L}}
\def\Lab{\La_{\rdb}}
\def\Lam{\La_{\rdm}}

\def\DP{D}
\def\DPc{\DP_{0}}
\def\DPb{\DP_{\rdb}}
\def\DPm{\DP_{\rdm}}

\def\LabGP{\La_{\rdb,\GP}}
\def\LamGP{\La_{\rdm,\GP}}

\def\DPbGP{\DP_{\rdb,\GP}}
\def\DPmGP{\DP_{\rdm,\GP}}
\def\riskbGP{\riskt_{\rdb,\GP}}

\def\gmi{\mathtt{b}}
\def\gmiid{\mathtt{g}_{1}}
\def\kullbi{\Bbbk}
\def\Thetasi{\Theta_{\loc}}
\def\rri{\mathtt{u}}
\def\rris{\rri_{0}}

\def\Ipc{\bb{\mathrm{f}}}
%\def\IF{\bb{\mathrm{f}}}
\def\IF{\Bbb{F}}
\def\IFc{\IF_{0}}
\def\IFb{\IF_{\rdb}}
\def\IFm{\IF_{\rdm}}


\def\DF{\cc{D}}
\def\DFc{\DF_{0}}
\def\DFb{\DF_{\rdb}}
\def\DFm{\breve{\DF}_{\rd}}
\def\DFm{\DF_{\rdm}}

\def\DPr{\breve{\DP}}
\def\VF{\cc{V}}
\def\VFc{\VF_{0}}

\def\HHc{\HH_{0}}
\def\HHb{\HH_{\rd}}
\def\HHm{\HH_{\rdm}}


\def\xib{\xi^{\flat}}
\def\xivb{\xiv_{\rdb}}
\def\xivm{\xiv_{\rdm}}
\def\CAm{\underline{\CA}}
\def\CAb{\CA}

\def\penr{\operatorname{pen}}
\def\pen{\mathfrak{t}}
\def\PEN{\operatorname{PEN}}
\def\RSS{\operatorname{RSS}}
\def\med{\operatorname{med}}

\def\ex{\mathrm{e}}
%\def\bracketing{\diamond}
\def\entrl{\mathbb{Q}}
%\def\entrlr{\mathbb{Q}^{\bracketing}}
\def\entrlb{\entrl}
%\def\entrlp{\entrl_{p}^{*}(\mrho)}
%\def\entrlq{\entrl_{p}^{*}(\qqq)}
%\def\entrlG{\entrl(\GV)}
\def\entr{\entrl}

\def\kullb{\cc{K}} %{\wp}
\def\kullbc{\kullb^{c}}


\def\gm{\mathtt{g}}
\def\gmc{\gm_{c}}
\def\gmb{\gm}
\def\gmbm{\gmb_{1}}

\def\yy{\mathtt{y}}
\def\yyc{\yy_{c}}
%\def\yyn{\yy_{0}}
\def\xx{\mathtt{x}}
\def\xxc{\xx_{c}}
\def\tc{t_{c}}

\def\alp{\alpha}
\def\alpn{\rho}
%\def\as{a_{0}}
%\def\daas{\Phi}
\def\gmu{\mathfrak{a}}


\def\losst{\varrho}
\def\loss{\wp}
\def\lossp{u}
\def\closs{g}

\def\riskt{\cc{R}}
\def\emprisk{\ell}
\def\bias{b}
\def\biasv{\bb{b}}
\def\bern{q}

%\def\nuu{\mathfrak{u}}
%\def\nud{\mathfrak{u}_{0}}
%\def\nun{c_{\nuu}}



\def\TT{\nsize}

\def\Pone{P}
%\def\Ef{\E}
%\def\Pf{\P}
\def\Pf{\P_{f(\cdot)}}
\def\Ef{\E_{f(\cdot)}}
\def\Ps{\P_{\thetas}}
\def\Es{\E_{\thetas}}
\def\Pu{\P_{\upsilons}}
\def\Eu{\E_{\upsilons}}

\def\Pvs{\P_{\thetavs}}
\def\Evs{\E_{\thetavs}}

%\def\upsdc{\ups_{0}}
\def\UPd{w}
\def\nunup{\nu_{1}}
\def\rru{\rr_{1}}
\def\rups{\rr_{0}}
\def\rupsb{\rups^{*}}
\def\rrf{\rr^{\flat}}
\def\rupd{\rr_{\circ}}


\def\smooths{\mathbb{S}}
\def\smooth{\smooths_{1}}


\def\elli{\bar{\ell}}


%\def\Pu{Q}

\def\K{K}

\def\Psir{\breve{\Psi}}

\def\af{a}
\def\afs{\af^{*}}

\def\kapla{\varkappa}

\newcommand{\mlew}[1]{\tilde{\thetav}_{#1}}
\newcommand{\mlea}[1]{\hat{\thetav}_{#1}}
\newcommand{\mluw}[1]{\tilde{\theta}_{#1}}
\newcommand{\mlua}[1]{\hat{\theta}_{#1}}
\newcommand{\penm}[1]{\boldsymbol{m}_{#1}}

\def\Pdom{\mu_{0}}
\def\PDOM{\bb{\mu}_{0}}
\def\EDOM{\E_{0}}

\def\mk{m}
\def\Mk{\cc{M}}
\def\SV{\cc{S}}

\def\Cs{E}
\def\Csd{\Cs^{\circ}}
\def\Ca{A}
\def\CS{\cc{E}}
\def\CA{\cc{A}}
\def\CAb{\CA_{\rd}}
\def\CAC{\CA_{\CoFu}}

\def\Ccb{m_{\rdb}}
\def\Ccm{m_{\rdm}}
\def\CcbGP{m_{\rdb,\GP}}
\def\CcmGP{m_{\rdm,\GP}}

\def\etas{\eta^{*}}

\def\omegav{\bb{\phi}}
\def\omegavs{\omegav^{*}}
\def\omegavc{\omegav'}

\def\nuvs{\nuv^{*}}
\def\nuvc{\nuv'}

\def\nunu{\nu_{0}}
\def\numu{\nu_{1}}
%\def\nubu{\nu_{2}}
\def\nupi{\nu^{+}}
\def\nubu{\beta}

\def\nus{\nu}
\def\nusb{\nus}
\def\nusr{\nus^{\bracketing}}
\def\Nusb{\mathbb{N}}
\def\Nusr{\mathbb{N}^{\diamond}}

\def\dist{d}
\def\distd{\mathfrak{a}}

\def\hatk{\kappa}
\def\ko{k^{\circ}}


%\def\qq{\mathfrak{q}}
\def\qqq{\mathfrak{q}}
%\def\ppp{\mathfrak{s}}
\def\ppp{{s}}
\def\Cqq{C(\qqq)}
\def\Cqqb{C^{\diamond}(\qqq)}
%\def\Cqq{\qqq \log(2 \ppp)}
\def\Crho{C(\mrho)}
\def\Cqqm{\log(4)}
\def\Cqpr{(\qqq \rrp + \dimp / 2)}

\def\Cdima{\mathfrak{e}_{0}}
\def\Cdimb{\mathfrak{e}_{1}}
\def\cdima{\mathfrak{c}_{0}}
\def\cdimb{\mathfrak{c}_{1}}
\def\cdim{\mathfrak{c}}

\def\rdomega{\varrho}
\def\deltaD{\delta}
\def\alphai{\alpha_{1}}
\def\alphaii{\alpha_{2}}
\def\alphaiii{\alpha_{3}}
\def\alphaiv{\alpha_{4}}

\def\err{\diamondsuit}
%\def\rd{\varrho}
%\def\errb{\err_{\rdomega}}
\def\errbm{\bar{\err}_{\rdomega}}
\def\errm{\err_{\rdm}}
\def\errb{\err_{\rdb}}


\def\errbGP{\err_{\rdomega,\GP}}
\def\errmGP{\err_{\rdm,\GP}}
\def\errbmGP{\bar{\err}_{\rd,\GP}}

\def\errs{\err_{\rdomega}^{*}}
\def\deltas{\alpha}

\def\xivbGP{\xiv_{\rdb,\GP}}
\def\xivmGP{\xiv_{\rdm,\GP}}


\def\SP{S}
\def\GP{G}
\def\GPt{\GP_{0}}
\def\GPn{\GP_{1}}
\def\gp{g}
\def\gs{s}

%\def\SP{G}

\def\errbGP{\err_{\rdb,\GP}}
\def\errmGP{\err_{\rdm,\GP}}
\def\errpmGP{\err_{\GP}^{\pm}}
%\def\errsGP{\err_{\GP}^{*}}
%\def\deltaDGP{\delta_{\GP}}

\def\LCS{\cc{C}}

\def\DPGP{\DP_{\GP}}
\def\thetavsGP{\thetavs_{\GP}}


\def\LL{\cc{L}}
\def\LLb{\LL^{*}}
\def\LLh{\cc{L}}

\def\YY{\cc{Y}}
\def\LP{L^{\circ}}


\def\modcnrd{\mathfrak{A}}

\def\pens{\pi}
\def\pnn{\mathfrak{g}}
\def\pnnd{\mathfrak{u}}
%\def\pnnd{b}
\def\pnndGP{\pnnd_{\GP}}


\def\confpr{\mathfrak{c}}
\def\confprb{\confpr^{*}}

\def\pn{\pens^{*}}
\def\penInt{\mathfrak{D}}
\def\penH{\mathbb{H}}
\def\pmu{\mathfrak{u}}
\def\Closs{\cc{R}}

\def\dimp{p}
\def\riskb{\riskt_{\rdb}}
%\def\dimpp{\mathfrak{p}}
\def\dimpp{\dimp+1}
\def\BB{I\!\!B}
\def\vA{\mathtt{v}}


\def\deficiency{\Delta}
\def\spread{\Delta}
\def\dimtotal{\dimp^{*}}

\def\thetav{\bb{\theta}}
%\def\thetavs{\thetav_{0}}
\def\thetavs{\thetav^{*}}
\def\thetavc{\thetav'}
\def\thetavd{\thetav^{\circ}}
\def\thetavdc{\thetav^{\sharp}}
%\def\dthetavs{\thetav,\thetavs}
%\def\dthetavc{\thetav,\thetavc}
%\def\dthetavd{\thetav,\thetavd}
\def\dthetavs{\thetav,\thetavs}

\def\thetas{\theta^{*}}
\def\thetac{\theta'}
\def\thetad{\theta^{\circ}}

\def\thetavb{\thetav^{\dag}}

\def\vtheta{\vartheta}
\def\vthetav{\bb{\vtheta}}
%\def\prior{\operatorname{pr}}
\def\prior{\Pi}

\def\Gam{\Xi}
\def\Gam{\mathcal{S}}
\def\RG{R}
\def\Psu{\Upsilon}
\def\Phim{\breve{\Phi}}

\def\Proj{P}

\def\gammavs{\gammav^{*}}
\def\gammavd{\gammav^{\circ}}
\def\etavs{\etav^{*}}
\def\etavd{\etav^{\circ}}
\def\etavc{\etav'}

\def\taus{\tau_{0}}
\def\taup{\lceil \tau \rceil}

%\def\Sigmas{{\Sigma^{*}}}
\def\sigmas{{\sigma^{*}}}
\def\Sigmas{\Sigma_{0}}

\def\upsilonc{\upsilon'}
\def\upsilond{\upsilon^{\circ}}
\def\upsilonp{{\upsilon}^{*}}
\def\upsilonm{{\upsilon}_{*}}
\def\upsilonvs{\upsilonv^{*}}
\def\upsilons{\upsilon^{*}}
\def\upsilonb{\bar{\upsilon}}
\def\upsilonvd{\upsilonv^{\circ}}

\def\ups{\bb{\upsilon}}
%\def\upss{\ups_{0}}
\def\upsc{\ups^{\prime}}
\def\upsd{\ups^{\circ}}
%\def\upsd{\mathring{\ups}}
%\def\upsd{\breve{\ups}}
\def\upsdc{\ups^{\sharp}}
\def\upsdu{\ups^{\flat}}
\def\upss{\ups^{*}}
\def\upsr{\breve{\ups}}

\def\Ups{\varUpsilon}
\def\Upsd{\Ups^{\circ}}
\def\Upss{\Ups_{\circ}}
\def\UpsP{\Ups^{c}}

\def\Thetas{\Theta_{0}}
\def\ThetasGP{\Theta_{0,\GP}}
\def\varthetav{\bb{\vartheta}}


\def\glink{g}


\def\fvs{\fv}
\def\fs{f}
\def\fb{\fv^{\dag}}



%\def\uu{\bb{u}}
\def\uc{\uv'}
\def\ud{\uv^{\circ}}
\def\uvs{\uv^{*}}
\def\us{u^{*}}
\def\vs{v^{*}}


\def\reps{\epsilon}
\def\eps{\epsilon}

\def\repsc{\reps_{0}}
\def\repsb{\reps^{*}}
%\def\repsg{\mathfrak{e}}
\def\repsg{g}

\def\lu{\delta}
\def\lub{\bar{\lu}}

\def\Uu{U}
\def\UU{\cc{Y}}
\def\UUM{\cc{M}}
\def\UP{\cc{U}}
\def\up{\mathfrak{u}}

\def\VP{V}
\def\VPc{\VP_{0}}
\def\VPV{\cc{U}}
\def\VPVc{\cc{\VPV}_{0}}
\def\VPGP{\VP_{\GP}}
\def\VPSP{\VP_{\SP}}

\def\VV{H}
\def\GV{\cc{G}}
\def\GVS{S}

\def\VVb{\VV^{*}}
\def\VVc{\VV_{0}}
\def\vv{\bb{h}}
\def\vva{g}
\def\vp{\mathbf{v}}
\def\vpc{\vp_{0}}
\def\VVca{\VV}
\def\Vtt{H}


\def\DG{D}


\def\Vn{V_{0}}
\def\vn{v_{0}}

\def\norm{\mathfrak{c}}
%\def\normc{\mathfrak{d}}
\def\normc{\delta}
\def\norma{c}

\def\egridd{\cc{E}_{\delta}}
\def\penb{\varkappa}

\def\dotzeta{\dot{\zeta}}
%\def\mes{\operatorname{mes}}
\def\mes{\pi}
\def\mesl{\Lambda}
\def\cprr{F}

%\def\lambdab{\bar{\lambda}}
\def\lambdam{\gm_{1}}
\def\lambdaB{{\lambda}^{*}}
\def\lambdac{{\lambda'}}

%\def\cla{\mathfrak{b}}
\def\cla{{b}}
\def\fis{\mathfrak{a}}
\def\fiss{\fis_{1}}

\def\Vd{{V}}
\def\vd{\bar{v}}

\def\klim{k^{\circ}}
\def\midm{\mid \!}

\def\Ldrift{M}
\def\ldrift{m}
\def\mY{b}
\def\Lvar{D}
\def\lvar{\sigma}

\def\Mubcu{\Upsilon}
\def\Dthetav{\bb{u}}


\def\B{\cc{B}}
%\def\BD{\mathring{\B}}
\def\BD{\B^{\circ}}
\def\BU{B}
\def\BI{\B^{*}}
%\def\dD{d^{*}}

%\def\Ns{\mathbb{N}}
%\def\Nsd{\mathbb{N}_{\thetavd}}

\def\mub{\mu^{*}}
\def\mubc{\mu}
\def\mubcb{\mubc^{*}}
\def\Mubc{\mathbb{M}}
\def\Mubcb{\mathrm{M}}

\def\zzc{\zz_{c}}
\def\ww{w}
\def\wwc{\ww_{c}}

\def\norms{\circ} %{\vartriangle}
\def\rs{\rr_{\norms}}
\def\yys{\yy_{\norms}}
\def\xxs{\xx_{\norms}}
\def\zzs{\zz_{\norms}}
\def\uu{\mathtt{u}}
\def\uus{\uu_{\norms}}
\def\mus{\mu_{\norms}}
\def\gms{\gm_{\norms}}
\def\wws{\ww_{\circ}}

\def\srho{s}
\def\mrho{\varrho}
%\def\mrhoc{\mrho'}

\def\Lmgf{\mathfrak{M}}
\def\Lmgfb{\Lmgf^{*}}
%\def\LMGF{\cc{M}}
%\def\LMGFu{\LMGF_{*}}
%\def\Lmgfd{\bar{\Lmgf}}
%\def\LmgfP{\Lmgf^{\circ}}

\def\lmgf{\mathfrak{m}}
\def\lmgfb{\lmgf^{*}}
%\def\lmgfd{\bar{\lmgf}}


\def\Expzeta{\mathfrak{N}}
\def\expzeta{\mathfrak{s}}

%\def\Lexpm{\mathfrak{S}}
%\def\expzetab{\expzeta_{0}}
%\def\ExpL{\cc{D}}
%\def\mUU{\mathfrak{b}}
%\def\ExpL{d}
%\def\dI{\ExpL^{*}}
%\def\ExpM{\cc{M}}
%\def\fz{f}



%\def\rough{R}

\def\rr{\mathtt{r}}
\def\rrb{\rr^{*}}
\def\rru{\rr_{\circ}}
\def\rrc{\rr'}
\def\rs{r_{*}}

\def\zz{\mathfrak{z}}
\def\zzb{\tilde{\zz}}
\def\tt{\mathfrak{t}}
\def\zb{z_{\rd}}
\def\zzg{\zz_{1}}
\def\zzQ{\zz_{0}}
\def\zzq{\zz}

\def\Cr{\mathfrak{c}}
\def\Crp{\mathfrak{C}}
\def\Crl{\mathfrak{r}}
\def\Crlp{\mathfrak{R}}
%\def\Crlq{\mathfrak{T}}
\def\Crlq{\cc{T}}
%\def\Crlqc{\Crlq_{0}}
\def\Crlmu{\cc{M}}





%%%%%%%%%%%%   semipar   %%%%%%%%%%%%%%%%%%%%%%%%
\def\zetah{\zeta_{h}}
\def\GG{G}
\def\HH{H}
\def\pG{p}
\def\pH{q}
\def\hh{H^{*}}

\def\mubch{\mubc_{1}}
\def\rhoh{\rho_{1}}
\def\CoFuh{\CoFu_{1}}
\def\dimh{p_{1}}
\def\VPh{\VP_{1}}
\def\VPt{\VP_{0}}

\def\LLh{L_{1}}
\def\pnndh{\pnnd_{1}}

\def\LCS{C}
\def\Ac{A_{0}}
\def\Ab{A_{\rd}}
\def\DPrb{\DPr_{\rdb}}
\def\DPrm{\DPr_{\rdm}}
%\def\zetavrb{\zetavr_{\rd}}
\def\Cb{\cc{C}_{\rdb}}
\def\Ub{\cc{U}_{\rdb}}
\def\zetavrb{\zetavr_{\rd}}
\def\xivrb{\breve{\xiv}_{\rd}}
\def\VPrb{\breve{\VP}_{\rdb}}
\def\Larb{\breve{\La}_{\rdb}}
\def\Larm{\breve{\La}_{\rdm}}

\def\deltav{\bb{\delta}}

\def\score{\nabla}
\def\scorer{\breve{\nabla}}

\def\LCS{C}
\def\Ac{A_{0}}
\def\Bc{B_{0}}
\def\AF{A}
\def\Ab{A_{\rdb}}
\def\Am{A_{\rdm}}
\def\DPrc{\DPr_{0}}
\def\DPrb{\DPr_{\rdb}}
\def\DPrm{\DPr_{\rdm}}
\def\Cb{\cc{C}_{\rdb}}
\def\Cm{\cc{C}_{\rdm}}
\def\Ub{\cc{U}_{\rdb}}
\def\deltav{\bb{\delta}}
\def\nuv{\bb{\nu}}
%\def\scorer_{\thetav}{\zetavr_{\rd}}
\def\xivrb{\breve{\xiv}_{\rd}}
\def\VPrb{\breve{\VP}_{\rdb}}
\def\Larb{\breve{\La}_{\rdb}}
\def\Lar{\breve{\La}}
\def\Larm{\breve{\La}_{\rdm}}
\def\VH{Q}
\def\VHc{\VH_{0}}
\def\zetavrm{\zetavr_{\rdm}}
\def\N{\mathbb{N}}

\def\Span{\operatorname{span}}
\def\Exc{{\square}}
\def\UUs{U_{\circ}}
\def\errbm{\errb^{*}}
\def\corrDF{\nu}
\def\BBr{\breve{\BB}}
\def\taua{\tau}
\def\AssId{\mathcal{I}}
\def\assId{\iota}
\def\AFD{\cc{A}}

\def\BanX{\cc{X}}
\def\basX{\ev}
\def\apprX{\alpha}
\def\fvs{\fv^{*}}
\def\lkh{\ell}
\def\Bc{B_{0}}
\def\dimn{\dimp_{\nsize}}
\def\betan{\beta_{\nsize}}


%%%%%%%%%%%   BvM   %%%%%%%%%%%%%%%%%%%%%%%%%%%%%%


\def\xivGP{\xiv_{\GP}}
\def\dimA{\mathtt{p}}
\def\dimAGP{\dimA}
\def\dime{\dimA_{e}}
\def\dimG{\dimA_{\GP}}
\def\dimS{\dimA_{s}}
\def\nubm{\nu_{\rd}}
\def\uub{u_{\rd}}
\def\uubGP{u_{\rd,\GP}}

\def\priorden{\pi}
\def\xivGP{\xiv_{\GP}}
\def\dimAGP{\dimA}
\def\nubm{\nu_{\rd}}
\def\uub{u_{\rd}}
\def\uubGP{u_{\rd,\GP}}

\def\CR{\mathcal{C}}
\def\CRb{\CR_{\rdb}}
\def\vthetavb{\bar{\vthetav}}
\def\Covpost{\mathfrak{S}}

\def\Db{\DP_{+}}
\def\Dm{\DP_{-}}
\def\uvb{\uv_{+}}
\def\uvm{\uv_{-}}
\def\uud{\omega}
\def\taub{\delta}
\def\Lip{L}
\def\Xb{X_{+}}
\def\Xm{X_{-}}
\def\deltam{\delta_{-}}
\def\betauv{\delta}
\def\betab{\betauv_{1}}
\def\betaf{\betauv_{2}}
\def\upsv{\bb{\varkappa}}
\def\upsvb{\bar{\upsv}}
\def\rhob{\varrho}
\def\alpb{\alp_{1}}
\def\betap{\betauv_{3}}
\def\Ec{\E^{\circ}}
\def\ff{f}
\def\fpos{g}
\def\fneg{h}
\def\alpb{\alp_{+}}
\def\alpm{\alp_{-}}




%%%%%%%%%%%   sms   %%%%%%%%%%%%%%%%%%%%%%%%%%%%%%
\def\kappak{\kappa}
\def\kappas{\kappak^{*}}
\def\Kappak{\cc{K}}
\def\DPk{\DP_{\kappak}}
\def\VPk{\VP_{\kappak}}



%%%%%%%%%%%%   sp   %%%%%%%%%%%%%%%%%%%%%%%%%%%%%
\def\ts{s}
\def\tsv{\bb{\ts}}
%\def\mm{\kappa}
\def\mm{\varkappa}
\def\mms{\mm^{*}}
\def\mmc{\mm'}
\def\mmd{\mm^{\circ}}
\def\mmo{\mm^{*}}
\def\mmmmo{\mm,\mmo}
\def\mmt{\tilde{\mm}}
\def\mma{\hat{\mm}}
\def\mmcs{\mm_{c}}
\def\mmcmm{\mmc,\mm}
\def\mmcmmd{\mmc,\mmd}
\def\mmset{\mathcal{I}}
\def\mmz{\mm_{0}}
\def\bmm{\breve{\mm}}

\def\qqmd{\qq^{\circ}}


\def\pp{z}

\def\LLL{L_{1}}
\def\LLr{L_{0}}
\def\muL{\mu_{1}}
\def\mur{\mu_{0}}

\def\LmgfL{\Lmgf_{1}}
\def\Lmgfr{\Lmgf_{0}}
\def\Lmgfm{\Lmgf_{1}}

\def\Kappa{\cc{K}}
\def\CoFu{\cc{C}}
\def\CoFuc{\CoFu_{0}}
\def\CoFub{\CoFu^{*}}
\def\CoFuL{\CoFu_{1}}
\def\CoFur{\CoFu_{0}}
\def\CAL{\CA_{1}}
\def\CAr{\CA_{0}}
\def\CAzz{\cc{A}}

\def\pnnL{\pnn_{1}}
\def\pnnr{\pnn_{0}}
\def\ttd{\delta}
\def\alphaL{\alpha_{1}}
\def\alphar{\alpha_{0}}
\def\alpharL{\alpha}
\def\rat{\mathfrak{t}}
\def\mquad{\nquad}
\def\zzL{\zz_{1}}
\def\zzr{\zz_{0}}

\def\xex{u}
\def\dcm{q}
\def\dc{g}
\def\dcL{\dc_{1}}
\def\dcr{\dc_{0}}
\def\kk{k}

\def\cpen{\tau}

%=================  density  ==============
\def\dens{f}
\def\jj{j}
\def\JJ{\cc{J}}
\def\Zphi{Z}
\def\Zphiv{\bb{\Zphi}}


%================= LES =====================
\def\nuu{\mathfrak{u}}
\def\nud{\mathfrak{u}_{0}}
\def\nun{c_{\nuu}}
\def\rhork{\kullb}
\def\GH{\mbox{GH}}
\def\HYP{\mbox{HYP}}
\def\NIG{\mbox{NIG}}
\def\IR{{\rm I\!R}}
\def\taggr{b}
\def\penm{\boldsymbol{m}}
\def\Crlp{\cc{R}}

%================== qfu/cmr ====================
\def\Mh{M}
%\def\Mht{\tilde{\Mh}}
\def\Mht{\Mh^{c}}

\def\Mhh{\Mh^{-}}
\def\Mhc{G}
\def\Lh{L_{1}}
\def\Uh{\cc{U}}
\def\wloc{w}
\def\Bias{B}
%\def\bias{b}
\def\ExpzetaU{\Expzeta_{1}}
\def\vpci{\vp_{i,0}}
\def\IFci{\IF_{i,0}}

\def\erqb{\Circle_{\rdb}}
\def\erqm{\Circle_{\rdm}}
\def\errqm{\errm^{*}}
\def\errqb{\errb^{*}}
\def\Nsize{N}
\def\VVD{\VV_{1}}
\def\AA{A}
\def\Wloc{W}

%%%%%%%%%%%   rough   %%%%%%%%%%%%%
\def\tups{\pen_{0}}
\def\rupd{\rr_{\circ}}
\def\VVb{\VVc}
\def\BP{B}
\def\bp{b}

\def\gps{s}
\def\GK{\cc{G}}

\def\zzGP{\zz_{\GP}}

\def\entrlq{\entrl_{1}}
\def\entrlg{\entrl_{2}}
\def\kb{k^{*}}


\def\rderr{\chi}
\def\Excgr{\diamondsuit}
\def\Excgrb{\diamondsuit^{*}}
%\def\thetavs{\thetav_{0}}
\def\Thetat{\bar{\Theta}}
\def\biasGP{\bias_{\GP}}
\def\QL{W}
\def\QLG{\mathcal{W}}
\def\BPGP{\QLG_{\GP}}
\def\BBGP{\BB_{\GP}}

\def\xxn{\xx_{\nsize}}
\def\fisGP{\mathtt{w}_{\GP}}
\def\risktGP{\riskt_{\GP}}

\def\dimq{q}
\def\nul{\mathrm{o}}
\def\Thetan{\Theta_{\nul}}
\def\thetavn{\thetav_{\nul}}
\def\thetavsn{\thetavs_{\nul}}
\def\tilden#1{\tilde{#1}_{\nul}}
\def\tildeGP#1{\tilde{#1}_{\GP}}
\def\xivn{\xiv_{\nul}}
\def\xivrGP{\xivr_{\GP}}
\def\DPcc{\DP_{\nul}}
\def\DPnGP{\DP_{1,\GP}}
\def\DPnGPr{\breve{\DP}_{1,\GP}}
\def\nablan{\nabla_{\nul}}
\def\scoren{\score_{\nul}}
\def\AnGP{A_{\nul,\GP}}

\def\testst{T}
\def\TGP{\testst_{\GP}}

\def\VPD{\VP_{2}}

\def\entrlB{\entrl_{1}}
\def\SB{W}
\def\dimq{q}
\def\QQ{\mathbb{H}}
\def\QQg{\QQ_{2}}
\def\QQq{\QQ_{1}}
\def\FF{F}
\def\LaGP{\La_{\GP}}

\def\zzQ{\zz_{\QQ}}
\def\zzAA{\zz_{\FF}}
\def\zzAAA{\zz_{\FF,\SB}}
\def\cdimc{\cdima}
\def\fisGP{\fis_{\GP}}
\def\rdomegab{\rdomega^{*}}

\def\lambdaGP{\lambda_{\GP}}
\def\wGP{\mathrm{w}_{\GP}}


\def\uudm{\mathtt{w}}
\def\lambdaB{\lambda_{\BB}}

\def\lambdav{\bb{\lambda}}
\def\etavd{\etav_{\circ}}
\def\thetavb{\breve{\thetav}}
\def\vthetavd{\Ec \vthetav}
%\def\vthetavd{\bar{\vthetav}_{\circ}}
\def\Covd{S_{\circ}}
\def\Covpostd{\Covpost_{\circ}}
\def\IS{\mathcal{I}}
\def\etas{\eta^{*}}
\def\Po{\operatorname{Po}}
\def\IF{\Bbb{F}}
\def\etavb{\bar{\etav}}
\def\etavd{\etav^{\circ}}
\def\Pc{\P^{\circ}}
\def\xxn{\xx_{\nsize}}
\def\CRd{\CR^{\circ}}

\def\CONST{\mathtt{C} \hspace{0.1em}}

\def\dimB{\mathtt{p}_{\BB}}

\def\nub{\nu}
\def\VPD{\VP_{2}}
\def\SB{W}
\def\dimq{q}
\def\dimqb{\dimq^{*}}
\def\QQ{\mathbb{H}}
\def\QQg{\QQ_{2}}
\def\QQq{\QQ_{1}}
\def\FF{F}

\def\qq{z}
\def\qqBB{\qq_{\BB}}
\def\qqQ{\qq_{\QQ}}
\def\qqAA{\qq_{\FF}}
\def\qqAAA{\qq_{\FF,\SB}}
\def\rderr{\chi}
\def\Excgr{\diamondsuit}

\def\Ccb{m}
\def\Ccm{m}
\def\BB{B}

\def\vthetavd{\vthetav^{\circ}}
\def\Indru{\Ind_{\rups}}

\def\BBh{U}
\def\DD{U}
\def\hsp{\tau}
\def\fiD{a}

\def\Prior{\Pi}
\def\prior{\pi}

\def\In{\mathcal{I}}
\def\KK{\cc{K}}
\def\qqu{\qq^{*}}
\def\ws{\omega}


\def\blk{\operatorname{block}}

\def\Xs{X^{*}}
\def\Xvs{\Xv^{*}}
\def\Xvg{\Xv_{\gammav}}
\def\Upsilons{\Upsilon_{0}}


\def\DPE{\mathrm{D}}
\def\DLS{\Lambda_{\sigma}}
\def\DFtt{\DF_{\thetav\thetav}}
\def\DFXX{\DF_{\etav\etav}}
\def\AtX{A_{\thetav\etav}}
\def\VPD{\VF_{2}}
\def\VX{\VP_{2}}
\def\vx{\mathtt{v}}
\def\vp{\mathrm{v}}
\def\dpe{\mathrm{d}}

\def\AB{B}

\def\Tau{\mathscr{T}}
\def\Taus{\Tau_{\sigma}}
\def\eigIF{\mathtt{f}}
\def\CNiGP{N_{\GP}^{(1)}}
\def\CNiiGP{N_{\GP}^{(2)}}
\def\CNsGP{N_{\GP}^{(s)}}

\def\CQX{\CONST_{\phi}}

\def\rupsGP{\rr_{\GP}}
\def\ExcGP{\err_{\GP}}

\def\GPc{\GP'}
\def\LGPc{L_{\GPc}}

\def\GPs{\GP^{*}}
\def\DPGPs{\DP_{\GPs}}
\def\xivGPs{\xiv_{\GP}^{\circ}}
\def\biasGPs{\bias_{\GPs}}
\def\GPF{\cc{\GP}}

\def\biasGPGPc{\bb{\bias}_{\GP,\GPc}}
\def\DPGPc{\DP_{\GPc}}
\def\DPDPc{\DP_{\GP,\GPc}}
\def\MM{\cc{X}}

\def\qqm{\qq^{-}}
\def\qqmu{\qqm_{2}}
\def\qqu{\qq^{+}}

\def\biasu{\beta}

\def\biasQL{\bias^{{\QL}}}
\def\GPGPc{\GP,\GPc}
\def\GPd{\GP^{\circ}}

\def\qqQL{\qq^{{\QL}}}
\def\qqQLu{\qq^{+}}
\def\qqmm{\qqm_{\mm}}

\def\Omegas{\Omega^{+}}
\def\Excgru{\Excgr^{+}}
\def\ExcGPu{\ExcGP^{+}}


\def\kkd{\kk^{\circ}}
\def\kkc{\kk'}
\def\kks{\kk^{*}}

\def\lossu{\loss^{+}}
\def\risku{\riskt^{+}}

\def\codd{\, | \,}
%\def\als{\alpha}
\def\als{a}
\def\Als{s}

\def\ws{\Lambda}

\def\MMplus{\MM^{2}_{+}}

\newcommand{\omegasn}[1]{\omega_{\dimp,#1}}

%%%%%%%%%%%%%%%%   GAR   %%%%%%%%%%%%%%%
\def\teps{\tilde{\varepsilon}}
\def\tepsv{\tilde{\epsv}}
\def\eps{\varepsilon}
\def\qqv{\bb{\qq}}
\def\tx{\tilde{x}}
\def\txv{\tilde{\xv}}
\def\txi{\tilde{\xi}}
\def\txiv{\tilde{\xiv}}
\def\tSv{\tilde{\Sv}}


%\def\betac{\beta_{1}}
\def\betac{b}
\def\Cfs{B}
\def\Meta{\mathbb{M}}
\def\Geta{\mathbb{G}}

\def\Wv{\bb{W}}
%\def\fm{s}
\def\fm{f}

\def\errXY{\, \Box}
\def\QXY{\mathbb{Q}}
\def\DeltaS{\Delta_{\Sigma}}

\def\ddv{U}
\def\deltan{\delta_{n}}

\def\up{u}
\def\Deltab{\bb{\Delta}}
%\def\fsp{\fs^{+}}
%\def\fsm{\fs^{-}}

\def\rrx{\rr_{x}}

\def\VPb{\cc{V}}
\def\QQb{\cc{H}}

\def\Cf{\mathtt{C}_{\fs}}
\def\sigml{\underline{\sigma}}
\def\sigmu{\bar{\sigma}}

\def\DS{\cc{G}}
\def\Deps{\Delta}
\def\qqY{\qq}
\def\qqz{z^{*}}
\def\Sigmab{\Sigma^{\sbt}}
\def\errxi{\Delta}
\def\errSi{\errXY}


\def\IFr{\breve{\IF}}
\def\Proa{\mathbb{H}_{1}}
\def\Pron{\mathbb{H}_{0}}
\def\bT{T^{\sbt}}
\def\bxivr{\breve{\xiv}^{\sbt}}

%%%%%%%%%%%%%%%%%%%%%%%%%%%%   bootstrap MS   %%%%%%%%%%%%%%%%%%%%%%%%%%%%
\def\sbt{\circ}

\def\Pb{\P_{\sbt}}
\def\Eb{\E_{\sbt}}
\def\bw{u}
%\def\bw{w^{\sbt}}
\def\bzeta{\zeta^{\sbt}}
\def\beps{\varepsilon^{\sbt}}
\def\berr{\err^{\sbt}}
\def\brhor{{\rhor^{\sbt}}}
\def\bxi{\xi^{\sbt}}
\def\bxiv{\xiv^{\sbt}}
\def\bSv{\Sv^{\sbt}}

%\def\mmmax{\bar{\mm}}
\def\mmmax{\mathtt{M}}
\def\md{\mm^{\circ}}
%\def\sbt{\, \sharp}
%\def\sbt{\#}
\def\sbt{\hspace{1pt} \flat}
%\def\sbt{\lozenge}%{\smallsquare}
\def\wb{u^{\sbt}}
\def\thetavbt{\tilde{\thetav}^{\sbt}}
\def\Tmdb{\Tmd^{\sbt}}
\def\xivb{\xiv^{\sbt}}
\def\zqb{\zq^{\sbt}}
\def\nB{B}
\def\Wb{\uv^{\sbt}}
\def\WB{\Uv^{\sbt}}
%\def\Ev{\bb{E}}
\def\epsvb{\epsv^{\sbt}}
\def\Varb{\Var_{\sbt}}
\def\Yvb{\Yv^{\sbt}}
\def\Yb{Y^{\sbt}}
\def\scoreb{\score^{\sbt}}
\def\VPb{{\VP^{\sbt}}}
\def\vB{\mathtt{v}}

%%%%%%%%%%%%%%%%%%%%%%%%%  single-index  %%%%%%%%%%%%%%%%%%%%%%%%%%%%%%

\def\xvd{\xv^{\circ}}
\def\yvd{\yv^{\circ}}
\def\gs{g}
%\def\IFone{F}
\def\IFone{\mathsf{H}}
\def\EPhi{\E_{\Phi}}
\def\Tv{\mathsf{T}}
\def\Av{\mathsf{A}}
\def\Tvs{\Tv_{0}}
\def\Phiv{\bb{\Phi}}
\def\PsiT{\mathsf{M}_{0}}
\def\PsiE{{\cc{M}}}
\def\psiM{\mathsf{m}}
\def\PsiV{{\mathsf{V}}}
\def\psiV{\mathsf{v}}
\def\usup{\mathsf{u}_{1}}
\def\Fr{\operatorname{Fr}}
%\def\CEDs{\mathsf{M}}
\def\ED{E\!D}
\def\Deltaf{\Delta_{\fv}}
\def\oper{\operatorname{op}}
\def\IFno{\IF_{0}}
\def\VPr{\breve{\VP}}
\def\VPno{\VP_{0}}
\def\vpno{\vp_{0}}
\def\fno{\mathsf{\vp}}
\def\Fno{\mathsf{H}}
\def\tups{\tau}
\def\tupsv{\bb{\tups}}
\def\VW{S}


\def\xxp{\xx_{\dimp}}
\def\xxnp{\xx_{n\dimp}}
\def\UV{\mathcal{U}}
\def\UVt{\mathcal{U}^{\T}}

\def\WP{W}
\def\MM{\cc{M}}
\def\MMu{\MM^{+}}
\def\MMd{\MM^{-}}
\def\MMi{\MM^{\circ}}
\def\NMMi{N^{\circ}}
\def\MMc{\MM^{c}}
\def\NMMc{N^{c}}
\def\msize{\mm}
%\def\mm{\mathbbmss{m}}
\def\mm{m}
\def\ms{\mm^{*}}
\def\mmb{\mm^{-}}
\def\mc{\mm'}
\def\mmmax{\mm_{\max}}
\def\jc{j'}
\def\md{\mm^{\circ}}
\def\xxm{\xx_{\msize}}
\def\xxb{\bar{\xx}}
\def\xxu{\xx^{c}}
\def\kku{\kk^{+}}
\def\kkm{\kk^{-}}
\def\kkb{\kk^{*}}
\def\qq{q}
\def\zzu{\zz^{+}}
\def\zzm{\zz^{-}}
\def\ZZu{\ZZ_{\alpb}}
\def\ZZm{\ZZ^{-}}
\def\ZZb{\bar{\ZZ}_{\alpb}}
\def\Deltau{\Delta^{+}}
\def\Deltam{\Delta^{-}}
\def\RMM{R}
\def\Tmmd{T}
\def\qqb{\qq^{*}}
%\def\ZZ{\mathtt{z}}
\def\ZZ{\mathtt{z}}
\def\ZZ{\mathfrak{z}}
\def\zq{z}
\def\zqb{\bar{\zq}}
\def\zND{z_{1}}
\def\zqu{\zq^{+}}
%\def\zpm{z^{\pm}}
\def\zpm{\zq^{\pm}}
\def\zqm{\zq^{-}}
\def\zqr{\mathtt{z}}
\def\RR{\cc{Q}}
\def\DQ{\cc{D}}
\def\VQ{\mathbb{V}}
\def\vq{\mathtt{s}}

\def\Tmd{\mathbb{T}}

%\def\dimq{\mathfrak{q}}
%\def\dimq{\mathtt{q}}
\def\dimq{q}
\def\alpb{\beta}
\def\alpd{\bar{\alp}}
\def\phis{\phi^{*}}
\def\biasm{b}
\def\riskr{\riskt^{+}}
\def\complx{\tau}
\def\ks{k^{*}}
\def\ZZb{\bar{\ZZ}}
\def\mmr{\breve{\mm}}

\def\thetavss{\thetav^{\star}}

%%%%%%%%%%%%%%%%%%%%%%%%%%%%   def   %%%%%%%%%%%%%%%%%%%%%%%%%%%%
%\def\mmmax{\bar{\mm}}
%\def\mmmax{\mathtt{M}}
%\def\sbt{\, \sharp}
%\def\sbt{\#}
\def\sbt{\hspace{1pt} \flat}
%\def\sbt{\lozenge}%{\smallsquare}
\def\thetavbt{\tilde{\thetav}^{\sbt}}
\def\Tmdb{\Tmd^{\sbt}}
\def\xivb{\xiv^{\sbt}}
\def\zqb{\zq^{\sbt}}
\def\nB{B}
\def\wb{w^{\sbt}}
%\def\wbr{\epsilon^{\sbt}}
\def\wbr{e^{\sbt}}
\def\Wb{\wv^{\sbt}}
\def\WB{\cc{W}^{\sbt}}
\def\WBr{\cc{E}^{\sbt}}
%\def\Ev{\bb{E}}
\def\epsvb{\epsv^{\sbt}}
\def\Varb{\Var_{\sbt}}
\def\Yvb{\Yv^{\sbt}}
\def\Yb{Y^{\sbt}}
\def\scoreb{\score^{\sbt}}
\def\VPb{{\VP^{\sbt}}}
\def\vB{\mathtt{v}}
\def\dimAb{\dimA^{\sbt}}
\def\maxPsii{\dPsi}
\def\XX{\cc{X}}
\def\AEB{\bb{A}}
\def\Bias{\bb{B}}
\def\sigmab{\bar{\sigma}}
\def\dPsi{\delta_{\Psi}}
\def\tdPsi{\dPsi}
\def\aPsi{a_{\Psi}}
\def\aSigma{a_{\Sigma}}
\def\PPsi{\mathbb{Q}}
\def\PPsib{\mathbb{Q}^{\sbt}}
\def\DeltaPsi{\Delta}
\def\GW{G}
\def\VPB{\cc{V}}
\def\Xvb{\Xv^{\sbt}}

\def\AGLM{A}
\def\AGLMb{\AGLM^{\sbt}}

\def\vpb{{\vp^{\sbt}}}
\def\zetavb{\zetav^{\sbt}}
\def\scorebr{\breve{\score}^{\sbt}}
\def\epsvr{\breve{\epsv}}
\def\epsr{\breve{\eps}}
\def\kk{m}

%%%%%%%%%%%%%%%%%%%%%%%%%%%%   end-def   %%%%%%%%%%%%%%%%%%%%%%%%%%%%
\def\pen{\mathtt{pen}}
\def\Sphere{\cc{S}}
\def\bvb{\bv^{\sbt}}
\def\bvs{\bv}
%\def\supA{\mathtt{a}}
\def\supA{\lambda}
\def\mmmin{\mm_{0}}
\def\mres{\breve{\mm}}
%\def\mmax{\mmmax}
\def\td{\delta}
\def\dimB{\dimA}
\def\vpB{\vp}
\def\lambdaB{\supA}
\def\zqc{\zq_{c}}


\def\totalpar{\phiv}
\def\totalpars{\totalpar^{*}}

\def\phivs{\phiv^{*}}
\def\nuisance{\phiv}
\def\nuisances{\phivs}
\def \nuisancesGP{\nuisances_{\GP}}
\def\upssGP{\upss_{\GP}}

\def\theteta{\upsilonv}
\def\thetavcut{\thetav,s}
\def\cutpar{\bb{\kappa}}
\def\cutpars{\cutpar^{*}}
\def\cutparkm{\cutpar}
\def\cutparskm{\cutpars}


\def\thetavss{\thetavs_{s}}
\def\thetavds{\thetavd_{s}}
\def\etavss{\etavs_{s}}
\def\etavskm{\etavs}
\def\upsm{\ups^{s}}
\def\thetetass{\thetetas_{s}}
%\def\thetavss{\thetav^{*s}}
%\def\thetavds{\thetav^{\circ s}}
%\def\etavss{\etav^{*s}}
%\def\thetetass{\theteta^{*s}}
\def\thetetas{\theteta^{*}}
\def\xivrs{\xivr_{s}}
\def\xivm{\xiv^{s}}

\def\thetavsm{\thetav^{s}}
\def\etavsm{\etav^{s}}
\def\etavkm{\etav}
\def\KM{\cc{S}}

%\def\xivrs{\xivr^{s}}
\def\biasc{\rho}
\def\biasp{b}
\def\ggdelta{\biasp_{\cutpar}}
\def\ggtheta{\biasc_{\cutpar}}
\def\km{m}
\def\ggdeltam{\biasp_{\km}}
\def\ggthetam{\biasc_{\km}}
\def\ggthetaGP{\biasc_{\GP}}
\def\ggdeltaGP{\biasp_{\GP}}
\def\VRF{\cc{E}}
\def\DPr{\breve{\DD}}
\def\DD{\mathbb{D}}
\def\DPs{\DP}
\def\DFs{\DF_{s}}
%\def\DPrs{\breve{\IFs}}
\def\DPrs{\mathbb{\DPs}}
\def\DFr{\breve{\DF}}


\def\CF{C}
\def\AP{A}
\def\AF{\cc{A}}


\def\alps{\alpha}
\def\rupc{\rr_{\cutpar}}
\def\rupm{\rr_{\km}}
%\def\scorers{\underline\scorer}
\def\AssIds{\AssId_{s}}
\def\scorers{\scorer}

%\def\IF{F}
\def\IFGP{\IF_{\GP}}
\def\DFGP{\DF_{\GP}}
\def\IFrm{\check{\IF}}

\def\IFt{\mathcal{F}}
\def\IFr{\IFt}
\def\IFs{\IFt_{\km}}
\def\IFsr{\IFt_{\km,\thetav}}


\def\xivrs{\breve{\xiv^{s}}}

%\def\AssIds{\AssId^{s}}
\def\EX{\E_{X}}
\def\block{\operatorname{block}}


\def\tar{\phi}
\def\tars{\tar^{*}}
\def\tarv{\bb{\tar}}
\def\tarvs{\tarv^{*}}

\def\Tam{\cc{K}}

\def\test{\tau}

\def\ZZs{\bar{\ZZ}}
\def\ZZbt{\ZZ^{\sbt}}
\def\qqbt{\qq^{\sbt}}
\def\dimAbt{\dimA^{\sbt}}
\def\xxt{t}
\def\Yr{\breve{Y}}
\def\Yvr{\breve{\Yv}}
\def\mres{\mm^{\dag}}
\def\Sam{\cc{T}}
\def\epsb{\eps^{\sbt}}
\def\Varxi{S}
\def\Varxib{S^{\sbt}}
\def\dPi{\td_{\Pi}}
\def\dimPi{\dimA_{\Pi}}

\def\rei{\delta}
\def\vpi{\mathtt{b}}
\def\dPsis{\delta^{*}}
\def\BBB{\cc{B}}

\def\Covm{\Sigma}
\def\covm{\sigma}
\def\supAB{\supA^{*}}

\def\supepsi{\delta_{\eps}}
\def\supeps{\delta_{1}}
\def \UVcol{\bb{\omega}}
\def\Covm{\mathbb{V}}

\input{statdef}

\title{
Multiscale parametric approach \\ for change point detection
}


\author{A. Suvorikova, V. Spokoiny, N. Buzun \\
\{suvorikova,spokoiny,buzun\}@wias-berlin.de}




\institute{
Weierstrass Institute for Applied Analysis and Stochastics
\and 
Institute for Information Transmission Problems 
}






\begin{document}

\maketitle
\begin{abstract}
This work presents a novel algorithm for change point detection, that can be applied for analysis of data of unknown nature. It is based on likelihood-ratio test statistics, as its behaviour can be described in terms of $\chi^2$-distribution even in case of model misspecification. To discover change point in the quickest way, statistics is calculated in a set of running windows of different scales. Algorithm is self-tuned: critical values are justified by data and calculated with multiplier bootstrap procedure. To make the method more robust for outliers, the concept of change-point patterns is presented.   
\end{abstract}

\begin{keywords} change point detection, multiscale inference
\end{keywords}

\newpage

\section{Introduction}
The problem of change point detection has a wide range of applications, that varies from life-critical to pure scientific ones. It appears each time one needs to explore a set of random data and make a decision about homogeneity of its structure. In other words, the problem can be stated as two following questions: were there any structural changes in the nature of observed data? At which moments, if so? These and similar questions arise in many areas of theoretical and engineering research. For example, algorithms of change point detection are used for identification and elimination of faults of aeroplane's navigation system, so as to perform better geolocation \citet{Nikif}. There are many other examples of real-world applications, such as analysis of stock markets \citet{Lavielle} or anomaly detection in computer traffic \citet{Tartak}, \citet{Casas}. This work mainly focuses on the \textit{sequential} or \textit{online} change point detection. In this case the data is aggregated from running random process. Let $Y_{\tau}$ be the observation at the current moment $\tau$, $\tau > 1$. 
The moment $\tau$ is a \textit{change point}, if stochastic properties of observed signal have undergone some changes: 
\[
\begin{cases}
Y_t \backsim \P_1 & t < \tau, \\ 
Y_t \backsim \P_2 & t \geq \tau. 
\end{cases} 
\]
%The goal is to detect regions of homogeneity, i.d. each $Y_t$ inside the region possess similar statistical properties. 
<<<<<<< HEAD
The goal is to find such structural brakes as soon as possible. 
This problem can be solved by using likelihood ratio test (LRT). This idea of application LRT for change point detection  goes back at least as far as \citet{quandt1960tests}. \citet{liu2008empirical}, \citet{zou2007empirical} investigate LRT for change point detection in nonparametric cases. In general, nonparametric approaches have greater delay in change point detection than their parametric alternatives. Introduction of \textit{parametric assumption}: $\P_1, \P_2 \in (\P(\theta), \theta \in \Theta \subseteq \R^p)$ allows to reduce average time of delay. The state-of-the-art review of parametric models based on LRT and its application to economics and bio-informatics are presented by \citet{ParStatChen}.  The paper \citet{gombay2000sequential} explores how it can be used for sequential change point detection in case $\P(\theta)$ is exponential family. \citet{lai1995sequential} proposes 'window-limited LRT schemes': test statistics is calculated in rolling window. This concept naturally expands to on-line change point detection. Many works are dedicated to asymptotic behaviour of LRT, e.g. \citet{jandhyala1999capturing} obtains lower and upper bounds for distribution of asymptotic maximum likelihood estimator. The work of \citet{kim1994tests} provides a very detailed study of its asymptotic behaviour in linear regression models. Similar results for a change in mean of a Gaussian process are in \citet{fotopoulos2010exact}. As far as the authors know, the most comprehensive study of the LRT behaviour is done by \citet{LRTWilks}. It shows that LRT is asymptotically $\chi^2$ distributed. The idea of the proof is based on the \textit{Wilk's phenomenon} \citet{wilks1938large}, \citet{boucheron2011high}. 
=======
The goal is to find such structural brakes as soon as possible. The problem arises across many scientific areas: quality control \citet{lai1995sequential}, cybersecurity   \citet{Cyber2}, econometrics \citet{SpokoinyCP}, \citet{Econom2}, geodesy e.t.c. Overview of the state-of-art methods of quickest change point detection are presented in \citet{ReviewPolun} or \citet{Shiryaev}.
The problem of change point detection can be easily reduced to the problem of hypothesis testing. Let $t'$ be a candidate for change point and let $(Y_{t' - h},..., Y_{t' + h - 1})$ be observed data in the rolling window of size $2h$, then
\[
H_0: Y_t \backsim \P_1, ~~t' - h \leq t \leq t' + h - 1
\]
\[
H_1: \begin{cases}
Y_t \backsim \P_1, & {t' - h } \leq t \leq {t' - 1}, \\ 
Y_t \backsim \P_2, & {t'} \leq t \leq t' + h - 1. 
\end{cases} 
\]
This problem can be solved by using likelihood ratio test (LRT). This idea of application LRT for change point detection  goes back at least as far as \citet{quandt1960tests}. It was proposed for detection of breaks in linear regression model. It was further developed by many authors, e.g. \citet{kim1989likelihood}, \citet{haccou1987likelihood}, \citet{srivastava1986likelihood}. \citet{liu2008empirical}, \citet{zou2007empirical} investigate LRT for change point detection in nonparametric cases. In general, nonparametric approaches have greater delay in change point detection than their parametric alternatives. Introduction of \textit{parametric assumption}: $\P_1, \P_2 \in (\P(\theta), \theta \in \Theta \subseteq \R^p)$ allows to reduce average time of delay. The state-of-the-art review of parametric models based on LRT and its application to economics and bio-informatics are presented by \citet{ParStatChen}.  The paper \citet{gombay2000sequential} explores how it can be used for sequential change point detection in case $\P(\theta)$ is exponential family. \citet{lai1995sequential} proposes 'window-limited LRT schemes': test statistics is calculated in rolling window. This concept naturally expands to on-line change point detection. Many works are dedicated to asymptotic behaviour of LRT, e.g. \citet{jandhyala1999capturing} obtains lower and upper bounds for distribution of asymptotic maximum likelihood estimator. The work of \citet{kim1994tests} provides a very detailed study of its asymptotic behaviour in linear regression models. Similar results for a change in mean of a Gaussian process are in \citet{fotopoulos2010exact}. As far as the authors know, the most comprehensive study of the LRT behaviour is done by \citet{LRTWilks}. It shows that LRT is asymptotically $\chi^2$ distributed. The idea of the proof is based on the \textit{Wilk's phenomenon} \citet{wilks1938large}, \citet{boucheron2011high}. 
>>>>>>> 10798e9edba0edf7cb1ee2af62aa206d065fb6af
The aim of the present paper is to described the LRT behaviour in finite-sample case using non-asymptotic Wilks and Fischer theorems \citet{wilks2013}. It is shown that the distribution of LRT is similar to ordinary $\chi^2$ under $H_0$. Otherwise if the sample is not homogeneous, the systematic drift of LRT appears. Thus, under $H_1$ the test statistic behaves like non-central $\chi^2$ random value. This drift is referred to as a \textit{change-point pattern}. The result holds for both correct and misspecified parametric models. 
The cornerstone of the new change point detection procedure is the concept of change-point pattern. The geometry of a pattern depends on a type of switch between distributions the data obeys before and after a change respectively. Three examples are presented at the Fig.~\ref{fig:patterns}. The triangle pattern appears in case of an abrupt switch from $\P_{\theta_1^*}$ to $\P_{\theta_2^*}$. A smooth transition between two regimes entails the trapezium change-point pattern. And an inverted triangle pattern appears due to a change in coefficients of linear regression. The control of a change-point pattern instead of a single LRT-value allows to reduce false-alarm rate to zero.

\begin{figure}[!h]
    \centering
    \includegraphics[width=0.9\textwidth, height=0.45\textwidth]{images/patterns-3.png}
    \caption{Type of change point and the geometry of change-point pattern}
    \label{fig:patterns}
\end{figure}

Any procedure of a change point detection requires information about the nature of the data observed after a structural break. The time of information aggregation is referred to as \textit{detection delay}. The proposed algorithm provides optimal \textit{detection delay} in the class of similar methods. The optimality is achieved by introduction of \textit{multiscale} approach. The technique is popular, e.g. \citet{multiscaleCP1}, \citet{SpokoinyCP} and allows analysis of the data using different scales simultaneously. The procedure proposed below computes the test statistics in rolling windows of different sizes and controls change-point pattern at each level. The greater number of scales at which a change-point pattern is detected, the more sure algorithm is, that there is a change point. 

Under some assumptions on the frequency of change points provided in Section~\ref{sec:theory}, the methods is applied to the \textit{multiple} change point problem. The survey on existing models can be found in \citet{chib98estimation}.
The last, but not at all the least feature of the proposed algorithm is that critical values for test statistic are computed in a data-driven way. The idea is to use the multiplier bootstrap procedure \citet{ChernozukovBoot}. The work of \citet{SpokoinyBoot}shows that it can be used for the construction of confides intervals even in case of a misspecified parametric model. Despite the fact, that theoretical properties of data-driven critical values are beyond the scope of this paper, the procedure of computation is presented in Algorithm~\ref{alg:bootstrap}.

The paper is organized as follows. Section~\ref{sec:procedure} presents the description of the algorithm. Theoretical properties of the procedure are discussed in Section~\ref{sec:theory}. Section~\ref{sec:experiments} compares the new algorithm with existing methods using simulated data sets. It also illustrates the performance of the method on a real data set. 

%To make the algorithm adaptive to change point of different sizes we introduce a . The LRT statistics is computed in rolling windows of different lengths. The approach and its modifications are quite popular in change point detection and for example is exploited in \citet{multiscaleCP1}, \citet{SpokoinyCP} and \citet{MultiscaleCP2}. The procedure relies on concept of multiscality is, for example, .Non-zero . It is referred to as The proof is based on novel algorithm of change point detection, that is described in the Section~\ref{sec:algorithm}.  an extension of \citet{LRTWilks} results to the finite-sample case. It shows that the behaviour of a fine sample is also $\chi^2$-like appropriate for multiple change point detection, e.g. \citet{multiscaleCP1}, \citet{chib1998estimation}. We assume change points \textcolor{red}{not to be very close to each other} and use \textit{multiscale} approach.  The similar idea of multiple testing for change \citet{SpokoinyCP} Thus, the introduction of \textit{multiscale} approach allows algorithm to be used for multiple Asumming that structural breaks are reasonably rare  frequency of the breaks %Nevertheless, it can be easily extended to retrospective frame work. The extension is presented in Section \ref{retrospective}.

%$Y_{i} \backsim \P_1$ if  and $Y_t \backsim \P_2$. 

%Let  $\mathbb{Y} = (Y_1, Y_2,..., Y_{\tau}, Y_{\tau + 1}...)$ be observed data. The goal is to detect a moment $\tau$ at which statistical properties of observed data change 

%Underlying assumption is that the structure of $\mathbb{Y}$ is homogeneous. It means that each $Y_i$ possess similar statistical properties. If the baseline assumption is wrong and the structure is non-homogeneous

%and $\mathbb{Y}$ is governed by some (presumably unknown) probability law $\P_1$.


%A change point is supposed to be detected if the hypothesis of homogeneity is rejected, in other words, there exists a moment $\tau$, $1 \leq \tau \leq N$, s.t. $(Y_1,..., Y_{\tau-1}) \backsim \P_1$ and $(Y_{\tau},..., Y_{N}) \backsim \P_2$, where $\P_2$ is not known as well. The moment $\tau$ is called a change-point. The goal is to find $\tau$ as precise as possible and minimise the number of false alarms and missed cases at the same time.
%

%\section{Related work}
%\label{sec:survey}
Methods that are used for change point detection can be classified in many different ways. Below we provide several standard ones. \\
\textit{Retrospective and sequential methods}.\\
This approach divides methods into two groups not by their properties, but by the area of their application.  Under \textit{retrospective} or \textit{offline} setting observed data set is fixed and the goal is to extract homogeneous regions. These methods are widely applicable for analysis of data that is not changing over time, e.g. images or DNA \citet{RecombBaeysian}. A very detailed survey of existing methods can be found in \citet{ParStatChen}. \textit{Sequential} or \textit{online} methods solve the problem of the \textit{quickest} change point detection. It is assumed, that the data is aggregated from running random process. The goal is to find changes in the nature of process as soon as possible. This problem arises across many scientific areas: quality control \citet{QualContr1}, cybersecurity \citet{Cyber1}, \citet{Cyber2}, econometrics \citet{SpokoinyCP}, \citet{Econom2}, geodesy e.t.c. Overview of the state-of-art methods for quickest change point detection are described in \citet{ReviewPolun} or \citet{Shiryaev}. 

\textit{Frequentist and Bayesian methods}.\\
\textit{Frequentist} approaches do not make any preliminary \textit{a prior} assumption about the stochastic nature of target parameter, i.d. it is supposed to be fixed value, not a random variable, e.g. \citet{Freq1}, \citet{Freq2}.
\textit{Bayesian} change point models, on the contrary, treat parameter as random variable, e.g. \citet{Bayes1}, \citet{Bayes2}. These methods are quite common in bio-statistics.

\textit{Parametric and non-parametric}.\\
All algorithms that assume observed data to obey some unknown stochastic law $\mathbb{P}_{\theta}$, that belongs to some known parametric family $(\mathbb{P}_{\theta}, \theta \in \Theta \subseteq \R^p)$ are called \textit{parametric}, e.g. \citet{Param1}, \citet{Param2}. Up-to-date survey of exiting methods and its applications can be found also in \citet{ParStatChen}.
\textit{Non-parametric} methods have more wide range of application, as they do not use any assumptions of this type, e.g.\citet{NonParam1}, \citet{NonParam2}. Many non-parametric methods can be found in \citet{NonParamRev}.

The concept of multiscality is, for example, exploited in \citet{multiscaleCP1}, \citet{SpokoinyCP} and \citet{MultiscaleCP2}. It means, that observed data is analysed on different scales simultaneously. In this work we broaden the idea of multiscale change point detection proposed in \citet{SpokoinyCP}.

As this realm of research is developing rapidly, more and more methods combine several of described techniques, e.g.  \textit{bayesian},  \textit{parametric} or \textit{non-parametric}  sequential change point detection \citet{BayesOnlineWeb}, \citet{RossCP}. There is a significant cohort of free soft-ware for researchers written in R and MatLab \citet{CPRepR}, \citet{BayesOnlineWeb}, \citet{GausMixtWeb}.



\onecolumn
\section{Algorithm}
\label{sec:procedure}
This section provides  the description of the proposed algorithm. Let $(\P(\theta), \theta \in \Theta \subseteq \mathbb{R}^p)$ be a local parametric assumption about the nature of data inside a window $(\YLL, \YRR)$. The generalised likelihood ratio test is 
\[
\dLh(t) = \sup_{\theta}L(\theta, Y_{t-h},..., Y_{t-1}) + \sup_{\theta \in \Theta}L(\theta; Y_{t},...,
Y_{t + h-1})
\]
\[
-\sup_{\theta \in \Theta}\{L(\theta; Y_{t-h},..., Y_{t-1}) + L(\theta; Y_{t},..., Y_{t+h-1})\},
\]
where $L(\theta;\cdot)$ is a log-likelihood function. 
To control a change point pattern, the procedure monitors $2h$ values of the LRT simultaneously:
\[
\dLhSeq(t) = \left(\sqrt{2 T_h(t - h)},\ldots, \sqrt{2 T_h(t + h - 1)}\right).
\]
The test statistics in hand is a convolution of $\mathbb{T}_h(t)$ with a predefined change-point pattern $P_h \in \R^{2h}$.

\[
    \dLhConv(t) = \left<\mathbb{T}_h(t), P_h\right>.
\]

Under \textit{online} framework, the algorithm marks a time moment $\tau$ as a change point,  if test statistics $\hat{T}_{h}(\tau + h)$ exceeds critical value $z_h$ at the moment $\tau + h$:
\[
\{\tau: \hat{\mathbb{T}}_{h}(\tau + h) > z_h\}.
\]

Under \textit{offline} setting, $\tau$ is a change point if
\[
\{\tau = \argmax_{t \in \{1,...,M\}} \hat{\mathbb{T}}_{h}(t)\},
\]
where $M$ is a time moment till which the data is observed.

In both cases the procedure repeats itself simultaneously on different scales $H = \{h_1,..., h_N\}$. The greater the number $k$ of such scales $h_{i_1},...,h_{i_k}$  where $\tau$ is marked as a change point, the more sure algorithm is, that $\tau$ the \textit{true} change point is.


%\label{sec:procedure}
This section provides  the description of the proposed algorithm. Let $(\P(\theta), \theta \in \Theta \subseteq \mathbb{R}^p)$ be a local parametric assumption about the nature of data inside a window $(\YLL, \YRR)$. The generalised likelihood ratio test is 
\[
\dLh(t) = \sup_{\theta}L(\theta, Y_{t-h},..., Y_{t-1}) + \sup_{\theta \in \Theta}L(\theta; Y_{t},...,
Y_{t + h-1})
\]
\[
-\sup_{\theta \in \Theta}\{L(\theta; Y_{t-h},..., Y_{t-1}) + L(\theta; Y_{t},..., Y_{t+h-1})\},
\]
where $L(\theta;\cdot)$ is a log-likelihood function. 
To control a change point pattern, the procedure monitors $2h$ values of the LRT simultaneously:
\[
\dLhSeq(t) = \left(\sqrt{2 T_h(t - h)},\ldots, \sqrt{2 T_h(t + h - 1)}\right).
\]
The test statistics in hand is a convolution of $\mathbb{T}_h(t)$ with a predefined change-point pattern $P_h \in \R^{2h}$.

\[
    \dLhConv(t) = \left<\mathbb{T}_h(t), P_h\right>.
\]

Under \textit{online} framework, the algorithm marks a time moment $\tau$ as a change point,  if test statistics $\hat{T}_{h}(\tau + h)$ exceeds critical value $z_h$ at the moment $\tau + h$:
\[
\{\tau: \hat{\mathbb{T}}_{h}(\tau + h) > z_h\}.
\]

Under \textit{offline} setting, $\tau$ is a change point if
\[
\{\tau = \argmax_{t \in \{1,...,M\}} \hat{\mathbb{T}}_{h}(t)\},
\]
where $M$ is a time moment till which the data is observed.

In both cases the procedure repeats itself simultaneously on different scales $H = \{h_1,..., h_N\}$. The greater the number $k$ of such scales $h_{i_1},...,h_{i_k}$  where $\tau$ is marked as a change point, the more sure algorithm is, that $\tau$ the \textit{true} change point is.


Algorithm~\ref{alg:alg_cp} summarises these ideas for sequential case. Here the current moment is supposed to be ${\tau + 2h_N - 2}$ and a candidate for the change point is $\tau$.

Algorithm~\ref{alg:bootstrap} presents the procedure of calculation of a critical value $z_h$ for a fixed window size $2h$. Let $\mathbb{Y} = (Y_1,..., Y_M)$ be a training set. Let weighted likelihood function be a convolution of i.i.d likelihood elements and a weight vector $(u_1,\ldots,u_M)$: 
\[
L^{\flat}(\theta; Y_1,\ldots, Y_M) = \sum_{m = 1}^{M} u_m l(\theta, Y_m),
\]
where $\{u_m\}_{m = 1}^M$ are i.i.d. and $\E u_m = \Var u_m = 1$. Then bootstrapped generalised likelihood ratio test is
\[
T_{h}^{\flat}(t) = \sup_{\theta \in \Theta}L^{\flat}(\theta; Y_{t-h},..., Y_{t-1}) + \sup_{\theta \in \Theta}L^{\flat}(\theta; Y_{t},..., Y_{t + h-1})
\]
\[
-\sup_{\theta \in \Theta}\{L^{\flat}(\theta; Y_{t-h},..., Y_{t-1}) + L^{\flat}(\theta; Y_{t},..., Y_{t+h-1})\},
\]

\begin{algorithm}[!th]
\label{alg:alg_cp}
\KwData{ $(Y_{\tau - 2h_N},...,Y_{\tau + 2h_N - 2})$, $\{h_1,..., h_N |h_1 < ... < h_N \}$, $\{z_{h_1},..., z_{h_N}\}$,  $\{P_{h_1},..., P_{h_N}\}$}
\KwResult{$\{I_1,..., I_N\}$. For all $k \leq N$, $I_k \in \{0, 1\}$ }

\For{$k \leftarrow 1$ \KwTo $N$}{
    Form $\mathbb{T}_{h_k}(\tau)$:
    \For{$l \leftarrow 1$ \KwTo $N$}{
        compute $T_{h_l}(\tau)$ using $(Y_{\tau - h_k},..., Y_{\tau + h_k - 1})$ \;
    }
 $\mathbb{T}_{h_k}(\tau) \leftarrow (T_{h_k}(\tau - h_k),..., T_{h_k}(\tau + h_k - 1))$\\
 $\hat{\mathbb{T}}_{h_k}(\tau) \leftarrow \sqrt{2\left<\mathbb{T}_{h_k}(\tau), P_{h_k}\right>}$\\
 
 \eIf{$\hat{\mathbb{T}}_{h_k}(\tau) > z_{h_k}$}{
        $I_k \leftarrow 1$ \;
     }{
        $I_k \leftarrow 0$\;
    }
}

\caption{Change point detection. $\tau$ is supposed to be a change point if exists at least one $I_k = 1$.}
\end{algorithm}


\begin{algorithm}[!th]
\label{alg:bootstrap}
\KwData{ $(Y_1,\ldots, Y_M)$,  $h$,  $P$, $S$ -- weights generation counts}
\KwResult{ $f^{\flat}_h$ -- bootstrap distribution of maximal convolution inside the window}

\For{$s = 1$ \KwTo $S$}{
  generate $u = (u_1,\ldots, u_M)$\;
  
  \For{each window position $t$}{
    compute $\dL_h^{\flat}(t)$\;
  }
  
  \For{each $\tau$}{
    $\hat{\mathbb{T}}^{\flat}(\tau) = \sqrt{2 \langle \mathbb{T}(\tau), P \rangle}$\;
  }
  add $\max_\tau\hat{\mathbb{T}}^{\flat}(\tau)$ to empirical distribution $f^b_h$\;
  
}

\KwData{$H = (h_1,\ldots,h_N)$, $f^b_h$ for each $h$, $\alpha$ -- confidence}
\KwResult{critical values $z_h$ for each $h$}

Multiplicity correction:\\

\For{$s$ = 1 \KwTo S}{
  generate $u = (u_1,\ldots, u_M)$\;
  
  add $\min_{h} \text{p-value}(\max_\tau\PT^b(\tau), f^b_h)$ to   empirical distribution $\P_f$  
  
}

find $\alpha'$ from condition $\P_f(x <  \alpha - \alpha') = \alpha$\;

\For{each $h$ in $H$}{
 $z(h) = $ quantile $(f^b_h, \alpha - \alpha')$\;
}

\caption{Critical values calibration}
\end{algorithm}





\onecolumn
\section{Theoretical results}
\label{sec:theory}
\subsection{LRT statistic}
%We are to derive essential properties of the Generalised Likelihood Ratio (GLR) statistic that form base of the CP detection algorithm. %Statements in this section rely on assumption 
This section presents main results that describe theoretical properties of the likelihood-ratio statistics (LRT). They are essential for the proposed algorithm of change point detection. Further assume that log-likelihood function $L(\theta) = L(Y,\theta)$ has rather precise approximation by its quadratic part in local region $\localr$  of $\theta^*$, $\localr \subseteq \R^p$, where
\[
\theta^* = \argmax_{\theta} \E L(\theta),
\quad
\widehat{\theta} = \argmax_{\theta} L(\theta)
\]
and $\localr = \{\Vert D (\theta - \theta^*) \Vert < r \}$. \cite{wilks2013} provides required conditions for justified quadratic approximation and parameter concentration in the local region.
Approximation error involves the next variables for its estimation:
\[
  \alpha(\theta, \theta_0) = L(\theta) - L(\theta_0)   - (\theta - \theta_0)^T \gradL( \theta_0) +  \frac{1}{2} \Vert D (\theta - \theta_0) \Vert^2, 
\]
\[
\chi(\theta, \theta_0) = D^{-1} \nabla \alpha(\theta, \theta_0) 
= D^{-1} (\gradL(\theta) - \gradL( \theta_0) ) +  D (\theta - \theta_0). 
\]
Let  in region $\Theta_0(r)$ with probability $1 - e^{-x}$:
\begin{equation}\label{cond_A}\tag{A}
\frac{| \alpha(\theta, \theta^*)  |}{\Vert D(\theta - \theta^*) \Vert} \leq \diamondsuit (r, x),  
\quad
  \Vert \chi(\theta, \theta^*) \Vert \leq  \diamondsuit (r, x),
\end{equation}
where $\diamondsuit (r, x) = (\delta (r) + 6 v_0 z_H(x) \omega ) r,$
\[\tag{D}
D^2(\theta) = - \nabla^2 \E L (\theta),
\quad
D = D(\theta^*),
\]
\begin{equation}\label{cond_dD}\tag{dD}
\Vert D^{-1} D^2(\theta) D^{-1} - I_p\Vert \leq \delta(r),
\end{equation}
\begin{equation}\label{cond_ED2}\tag{ED2}
\forall \lambda \leq g, \; \gamma_1 \gamma_2 \in \R^p: \quad
\log \E \exp \left\{
\frac{\lambda}{\omega} \frac{\gamma_1^T \nabla^2 \overset{o}{L}(\theta) \gamma_2}{\Vert D \gamma_2 \Vert \Vert D \gamma_2 \Vert}
\right\} \leq 
\frac{\nu_0^2 \lambda^2}{2},
\end{equation}
\[
z_H(x) = \sqrt{H} + \sqrt{2x} + \frac{g^{-2} x + 1}{g} H, 
\quad H = 6p.
\]
Condition (\ref{cond_dD}) ensures quadratic approximation of $\E L(\theta)$ and (\ref{cond_ED2}) ensures linear approximation of centered likelihood $\overset{o}{L}(\theta) = L(\theta) - \E L(\theta)$.  

Firstly, to provide a simple non-strict explanation of what kind of distribution the main statistic  $\dLh$ is supposed to have, review  $\dLh$ as
\[
\dLh = L(\widehat{\theta}) -  L(\widehat{\theta}_{H_0}), 
\quad L(\theta) = L_1(\theta_1) + L_2(\theta_2), 
\]
\[
\quad L_1 = L(Y_1,\ldots,Y_h), \; L_2 = L(Y_h,\ldots,Y_{2h}), 
\]
where $\widehat{\theta}_{H_0}$ is argmax of $L$ under condition $H_0: \theta_1^* = \theta_2^*$. Then due to quadratic approximation $\dLh$  corresponds to Tailor equation with point $\widehat{\theta}$:
\[
\dLh \approx \frac{1}{2} \Vert D(\widehat{\theta} - \widehat{\theta}_{H_0})  \Vert^2.
\]   
If $\widehat{\theta}$ and $\widehat{\theta}_{H_0}$ tend to be Normal and $H_0$ is true then their difference are close to a centered Normal variable. If $H_0$ is false -- the Normal variable will have mean that  is equal to $D(\theta^* - \theta_{H_0}^*)$. 

The next equations describes strict equation for LRT statistic distribution in  quadratic model case. 
\begin{align*}
L(\theta) &= L_1(\theta) + L_2(\theta) \\
&=  
L_1(\widehat{\theta}_1) + L_2(\widehat{\theta}_2) -  \frac{1}{2} (\theta - \widehat{\theta}_1)^T D_1^2  (\theta - \widehat{\theta}_1)
- 
\frac{1}{2} (\theta - \widehat{\theta}_2)^T D_2^2  (\theta - \widehat{\theta}_2)  \\
 &=
 L(\widehat{\theta}) - \frac{1}{2} (\theta - \widehat{\theta})^T D^2  (\theta - \widehat{\theta}),
\end{align*}
\[
\widehat{\theta} = D^{-2} (D_1^2 \widehat{\theta}_1 + D_2^2 \widehat{\theta}_2 ),
\quad D^2 = D_1^2 + D_2^2.
\]    
\begin{align*}
\dLh &=
L_1(\widehat{\theta}_1) + L_2(\widehat{\theta}_2) -  L(\widehat{\theta}) \\
&=   \frac{1}{2} (\widehat{\theta} - \widehat{\theta}_1)^T D_1^2  (\widehat{\theta} - \widehat{\theta}_1)
+ \frac{1}{2} (\widehat{\theta} - \widehat{\theta}_2)^T D_2^2  (\widehat{\theta} - \widehat{\theta}_2).
\end{align*}
\[
\widehat{\theta} - \widehat{\theta}_1 = D^{-2} (D_1^2 \widehat{\theta}_1 + D_2^2 \widehat{\theta}_2 ) - \widehat{\theta}_1 = D^{-2} D_2^2 ( \widehat{\theta}_2 -  \widehat{\theta}_1),
\]
\[
\widehat{\theta} - \widehat{\theta}_2 = D^{-2} (D_1^2 \widehat{\theta}_1 + D_2^2 \widehat{\theta}_2 ) - \widehat{\theta}_2 = D^{-2} D_1^2 ( \widehat{\theta}_1 -  \widehat{\theta}_2).
\]
\[
2 \dLh =  ( \widehat{\theta}_2 -  \widehat{\theta}_1)^T \Sigma^2 ( \widehat{\theta}_2 -  \widehat{\theta}_1),
\]
where
\[\tag{S}
\Sigma^2 = D_2^2 D^{-2} D_1^2 D^{-2} D_2^2 + D_1^2 D^{-2} D_2^2 D^{-2} D_1^2
 = D_1^2 D^{-2}D_2^2 \approx \frac{1}{4} D^2. 
\]
In quadratic model  following equations enables replacement of $\widehat{\theta}_2$,  $\widehat{\theta}_1$ in the equation for $\dLh$ with regard to condition $\chi(\theta, \theta^*) = 0$:
\[
 D_1(\widehat{\theta}_1 - \theta_1^*) = \xi_1 = D_1^{-1} \nabla L(\theta_1^*), \quad
D_2(\widehat{\theta}_2 - \theta_2^*) = 
\xi_2 = D_2^{-1} \nabla L(\theta_2^*).
\]
The next theorem concludes these considerations to  generalized result for non-quadratic model. 
\begin{theorem}
\label{dl_theorem}
Assume condition (\ref{cond_L_star}) and quadratic Laplace approximation (\ref{cond_A}) of $L_1$ and $L_2$   are fulfilled with probability $1 - 2 e^{-x}$, additionally with probability $1 - 2 e^{-x}$
\[
\Vert \xi_i \Vert \leq z(x), 
\quad z^2(x) = \max_i p_{B_i} + 6 \lambda_{B_i} x,
\] 
\[\tag{B}
B_i = D_i^{-1} \Var (\nabla L_i(\theta^*))D_i^{-1},
\quad p_B = \tr(B), 
\quad \lambda_B  = \lambda_{\max} (B).
\]
Then in the local region with probability $1 - 8 e^{-x}$ 
\[
2 \dLh = \Vert  \dxi + \dtheta \Vert^2  + O(\{r + z(x)\} \diamondsuit (r, x)),
\]
where
\[
\dxi  = \Sigma (D_2^{-1} \xi_2 - D_1^{-1} \xi_1),
\quad
\dtheta  = \Sigma (\theta_2^* - \theta_1^*).
\]
\end{theorem}    

\begin{remark} 
\label{dxi_limit}
In increasing sample size $n \to \infty$ the stochastic component tends to Normal distribution: 
\[
\dxi \to \mathcal{N}(0, B_1 + B_2).
\]
\end{remark}

\begin{remark}
For the condition $\widehat{\theta} \in \Theta_1(r) \cap \Theta_2(r)$ the restriction of the parameter  $\theta^*$  variability is required
\begin{equation}\label{cond_L_star}
\tag{L*}
\Vert D(\theta_1^* - \theta_2^*) \Vert \leq r.
\end{equation}
\end{remark}

Proof of  a similar statement (theorem \ref{dl_theorem}) for statistic $\sqrt{2 \dLh}$ one could get from condition (\ref{cond_A}). With probability $1 - 2e^{-x}$
\begin{align*}
\left| \dLh (\widehat{\theta}_1, \widehat{\theta}_2) - \frac{1}{2} \Vert \Sigma (\widehat{\theta}_2 -\widehat{\theta}_1) \Vert^2 \right|  
&\leq
2 \Vert D_1(\widehat{\theta}_1 - \widehat{\theta}) \Vert \diamondsuit (r, x) + 2 \Vert D_2(\widehat{\theta}_2 - \widehat{\theta}) \Vert \diamondsuit (r, x) \\ 
&\leq  4  \Vert  \Sigma(\widehat{\theta}_2 - \widehat{\theta}_1)  \Vert  \diamondsuit (r, x).
\end{align*}
Inequality $|a - b| \leq |a^2 - b^2| / b, \; b >0$ converts the previous term to
\[
\left| \sqrt{ 2  \dLh (\widehat{\theta}_1, \widehat{\theta}_2) } -  \Vert \Sigma (\widehat{\theta}_2 -\widehat{\theta}_1) \Vert \right| \leq 
8   \diamondsuit (r, x).
\]
Replacement $(\widehat{\theta}_1, \widehat{\theta}_2)$ with $(D_1^{-1}\xi_1 + \theta_1^*, \; D_2^{-1}\xi_2 + \theta_2^*)$ results in
\[
\left| \Vert \Sigma (\widehat{\theta}_2 -\widehat{\theta}_1) \Vert  - 
\Vert  \dxi +  \dtheta \Vert 
\right | 
\]
\[
\leq  \Vert \Sigma(\widehat{\theta}_1 - \theta_1^*) - \Sigma D_1^{-1} \xi_1 \Vert
+ \Vert \Sigma(\widehat{\theta}_2 - \theta_2^*) - \Sigma D_2^{-1} \xi_2 \Vert
\leq  2 \diamondsuit (r, x).
\]
The next theorem summarizes the statements above.  
\begin{theorem}
\label{dl_sq_theorem}
Assume condition (\ref{cond_L_star}) and quadratic Laplace approximation (\ref{cond_A}) with probability $1 - 2 e^{-x}$ are fulfilled. Then  with probability $1 - 4 e^{-x}$ in the local region  $\Theta_1(r) \cap \Theta_2(r)$ took place
\[
\left| 
\sqrt{ 2\dLh} - 
\Vert \dxi + \dtheta \Vert 
\right| \leq 
10  \diamondsuit (r, x).
\]
where $\dxi$ and $\dtheta$ are defined in theorem \ref{dl_theorem}.
\end{theorem}

\begin{remark}
 The constant near  $\diamondsuit (r, x)$ could be decreased,  expanding  series of $L_1(\theta)$, $L_2(\theta)$ and $L(\theta)$ in the local regions around $\theta_1^*$, $\theta_2^*$ and $\theta^*$ instead of MLE values:
\begin{align*}
2 \dLh &= - \Vert \xi \Vert^2 + \Vert \xi_1 \Vert^2 + \Vert \xi_2 \Vert^2  
-2 \xi_1^T D_1 D^{-2} D_2^2 (\theta_2^* - \theta_1^*) 
+ 2 \xi_2^T D_2 D^{-2} D_1^2 (\theta_2^* - \theta_1^*) \\
&\quad+ \Vert  D_1 D^{-2} D_2^2 (\theta_2^* - \theta_1^* )\Vert^2 + \Vert  D_2 D^{-2} D_1^2 (\theta_2^* - \theta_1^*) \Vert^2 
\pm (2 \diamondsuit (r, x) r + 2 \delta(r) r^2) \\
&= - \Vert \xi \Vert^2 + \Vert \xi_1 \Vert^2 + \Vert \xi_2 \Vert^2   
+2(D_2^{-1} \xi_2 - D_1^{-1} \xi_1)^T \Sigma^2 (\theta_2^* - \theta_1^*) + \Vert \Sigma (\theta_2^* - \theta_1^*) \Vert^2 \\
&\quad\pm (2 \diamondsuit (r, x) r + 2 \delta(r) r^2).
\end{align*}
Referring to condition~\ref{cond_A}, $\Vert D^{-1}(D_1 \xi_1 + D_2 \xi_2) \Vert^2 \pm  2\diamondsuit (r, x) z(x) $ replaces $\Vert \xi \Vert^2$. 
\[
- \Vert \xi \Vert^2 + \Vert \xi_1 \Vert^2 + \Vert \xi_2 \Vert^2  
=\Vert \Sigma (D_2^{-1} \xi_2 - D_1^{-1} \xi_1) \Vert^2 \pm  2\diamondsuit (r, x) z(x).
\]
That leads to result
\[
\bigg |
2 \dLh  -  \Vert  \dxi +  \dtheta \Vert^2 
\bigg | 
\leq (4 \diamondsuit (r, x) r + 2 \delta(r) r^2).
\]
\end{remark}

\begin{remark}
Weighted LRT statistic (\ref{Tb}) has  similar approximation:
\[
2 \dLh^{\flat} \approx  \Vert D(\widehat{\theta}^b - \widehat{\theta}_{H_0}^b)  \Vert^2 =  \Vert \dxi^{\flat}  \Vert^2.
\] 
where $\widehat{\theta}^{\flat}_{H_0}$ is argmax of $L^{\flat}$ under condition $H_0: \widehat{\theta}_2 - \widehat{\theta}_1 = \widehat{\theta}_{12}$, which is true. That's why the mean of difference $(\widehat{\theta}^b - \widehat{\theta}_{H_0}^b)$  is zero.
\end{remark}

\subsection{Optimal window size}
\label{subsec:win_size}
The change point detection algorithm described above has rather meaningful parameter window size ($h$) that determines sample sizes on which MLE ($\widehat{\theta}_1$, $\widehat{\theta}_2$) will be compared. It is possible to  find out the minimal required sample size from condition 
\[
h\KL (\theta_1^*, \theta_2^*) > h\KL (\widehat{\theta}_1, \theta_1^*) +
h\KL (\widehat{\theta}_2, \theta_2^*).
\]
Wilks theorem (reg. \cite{wilks2013}) gives upper approximation with probability $1-10e^{-x}$ 
\[
h \KL (\widehat{\theta}_1, \theta_1^*) +
h \KL (\widehat{\theta}_2, \theta_2^*) \leq 
2 r \diamondsuit(r,x) + \frac{\Vert \xi_1 \Vert^2}{2} + \frac{\Vert \xi_2 \Vert^2}{2},
\]
where with probability $1-4e^{-x}$
\[
\frac{\Vert \xi_1 \Vert^2}{2} + \frac{\Vert \xi_2 \Vert^2}{2} \leq z^2(x)  = p_B + 6 \lambda_B x,
\]
In case with 
\[
r \diamondsuit(r,x) = \sqrt{\frac{C(p_B + x)^3}{h}},
\quad h > C(p_B+x),
\]
lower bound for parameter change results in estimation
\[
h \KL (\theta_1^*, \theta_2^*) > 3 p_B + (6 \lambda_B + 2) x.
\]


Optimal $h$ is finite. Increasing a sample size one decreases an impact of stochastic part of $\Vert  \dxi +  \dtheta \Vert$, since  $\Vert  \dtheta \Vert$ grows. But at the same time $\Vert  \dtheta \Vert$ will not be changed by window replacement when $h \to \infty$. Note also that angle of $\Vert  \dtheta \Vert$ growth decreases with $h$, so the optimal window size is the smallest one that is sufficient to overcome random fluctuations in convolution of $\Vert \dxi(i) + \dtheta(i) \Vert$ with linear function $f(i) = i$.   
Define new variables
\[
b = \Vert  \dtheta \Vert = \sqrt{h} b_0,
\quad 
b_i = \frac{i}{h} b, \; i > 0,
\quad
\xi_i =  \dxi(i).
\]
Optimal window size for online change point detection is to be derived from the following inequality.
\[
\sum_{i = 1}^{h} i \Vert \xi_i + b_i \Vert \geq \sum_{i = 1}^{h} i \bigg(\Vert \xi_i \Vert + 10 \diamondsuit(r,x) \bigg). 
\]
Theorem 4.1 from paper \cite{paqf2013} ensures following inequality with probability $1 - 2 e^{-x}$ 
\[
\Vert \xi_i + b_i \Vert \geq \sqrt{ \Vert \xi_i \Vert^2 + \Vert b_i \Vert^2  - 2 \Vert b_i \Vert - 2 \delta_1(x) } 
\geq 
\]
\[
\geq \Vert b_i \Vert - 2  - \sqrt{4 + 2 \delta_1(x)}.
\]
 With probability $1 - 4 e^{-x}$ under condition that statement from theorem \ref{dl_sq_theorem} is true one  comes to a final estimation of the minimal sufficient window size: 
\[
h \geq \frac{9 (2 + \sqrt{4 + 2 \delta_1(x)} + z(x) +  10 \diamondsuit(r,x) )^2 }{4 b_0^2} \sim \frac{c_1 + c_2 p}{b_0^2}. 
\] 

\onecolumn
\section{Experiments}
\label{sec:experiments}

\subsection{Experiments with synthetic data}

This section presents results of the comparison of the proposed algorithm of change point detection (referred as \textit{LRTOnline} or \textit{LRTOffline}) with two other methods: \textit{Bayesian online changepoint detection (BOCPD)}  \citet{BayesOnlineWeb} and \textit{cpt.meanvar(PELT,$\ldots$)} (RMeanVar) from \citet{RPackage}. 
The first method is constructed for online inference, but so far as it returns CP location with each CP signal, it is also applicable for offline testing scenario. The idea of this method is predictive filtering: its forecasts a new data point using only the information have been observed already, where the distribution family is fixed (Normal for the tests in this paper). Bayesian inference calculates the length of the observed data (from the last CP).
The second algorithm also uses preliminary specified model. Its design focuses into finding multiple changes in mean and variance in Normally (another distributions also supported) distributed data. The returned set of change points is the result of sequential testing $H_0$ (existing number of change points) against $H_1$ (one extra change point) applying the likelihood ratio statistic of the whole data coupled with the penalty for CP count. Originally the method has offline change point detection interface, but one could adapt it for online case by buffering incoming data elements and clearing the buffer when at least one CP have been observed in the buffered data. 
In total, each of these two algorithms has modification in the way that allows one to use it in both online and offline testing mode. 

LRTOffline configuration:  \\
window sizes $(h_1, \ldots, h_W) = (10, 20, 40, 70)$; 
confidence for the upper bound of convolution with pattern $= 0.1$; 
window weights $(u_1, \ldots, u_W)  = (1.0, 2.0, 0.5, 0.2)$. 

LRTOnline configuration:  \\
window sizes $(h_1, \ldots, h_W) = (30, 50, 70)$; 
confidence $= 0.1$.


Quality of measurements  uses three following metrics: Normalised Mutual Information (NMI), Delay (average time interval in which CP have been detected after it had taken place), Precision  and Recall. The next equation defines 
NMI measure of two partitions ($X$, $Y$) of time range by change points
\[
\text{NMI}(X,Y) = 2 \frac{H(X) + H(Y) - H(X,Y)}{H(X) + H(Y)}.
\]
Higher NMI values (they are in $[0,1]$) correspond to better quality. 
Quality comparison in offline case apply NMI measure, while  for online mode involves Delay, Precision  and Recall. 

Synthetic test data have been generated for different values of difference norm of the data distribution parameter. Such values are denoted as \textit{delta}. Each delta corresponds to 10 sampled data sequences over which one compute measure average.  In online mode each data sequence could have one or none change points,  in offline mode -- two, one or none change points.       

\begin{figure}[ht!]
     \begin{center}
        \subfigure{%
            \label{fig:Precision_delta_N}
            \includegraphics[width=0.4\textwidth]{images/NMI_delta_N}
        }%
        \subfigure{%
           \label{fig:Recall_delta_N}
           \includegraphics[width=0.4\textwidth]{images/NMI_delta_Po}
        }
         
         
%
    \end{center}
    \caption{%
        Offline mode. First data: $\ND(\theta(1), \theta(2))$, second data: $\Po(\theta)$,
        delta = $\Vert \theta_{12}^{*} \Vert$, data size = 340, PA for all methods is $\ND(\theta(1), \theta(2))$, NMI -- Normalized Mutual Information between predicted and reference partitions of time interval with change points, tests per delta = 10, change point per test = $\{0,1,2\}$.
     }%
   \label{fig:subfigures}
\end{figure}


\begin{figure}[ht!]
     \begin{center}
        \subfigure{%
            \label{fig:Precision_delta_N}
            \includegraphics[width=0.3\textwidth]{images/Precision_delta_N}
        }%
        \subfigure{%
           \label{fig:Recall_delta_N}
           \includegraphics[width=0.3\textwidth]{images/Recall_delta_N}
        }
         \subfigure{%
           \label{fig:Delay_delta_N}
           \includegraphics[width=0.3\textwidth]{images/Delay_delta_N}
        }\\%  ------- End of the first row ----------------------%
          \subfigure{%
            \label{fig:Precision_delta_N}
            \includegraphics[width=0.3\textwidth]{images/Precision_delta_Po}
        }%
        \subfigure{%
           \label{fig:Recall_delta_N}
           \includegraphics[width=0.3\textwidth]{images/Recall_delta_Po}
        }
         \subfigure{%
           \label{fig:Delay_delta_N}
           \includegraphics[width=0.3\textwidth]{images/Delay_delta_Po}
        }
%
    \end{center}
    \caption{%
       Online mode. First row data: $\ND(\theta(1), \theta(2))$, second row data: $\Po(\theta)$,
        delta = $\Vert \theta_{12}^{*} \Vert$, data size = 340, PA for all methods is $\ND(\theta(1), \theta(2))$, tests per delta = 10, change point per test = $\{0,1\}$.
     }%
   \label{fig:subfigures}
\end{figure}

In the offline tests with Normal data all the methods achieves similar NMI scores, nonetheless LRTOffline is more stable for decreasing strength of CP (delta). In the tests with Poisson data (misspecification) RMeanVar has relatively low quality. The online tests characterizes the proposed method (LRTOnline) as more stable along different delta values what is accomplished by multiscale heuristic.      

 The experiments reveal following meaningful properties of the proposed method configuration:

\begin{enumerate}
\item Quality is sensitive to selection of interval for upper bound calibration of convolution in offline mode. For example in data $\ND(0,1)$.sample(100) $\cup$ $\ND(1,2)$.sample(100) is preferable to use only slice of 0 to 100 elements for calibration, because of lower $\Var \dxi$.  Generally one should find change points in $tr(B_1 + B_2)$ according to remark~\ref{dxi_limit} from Section \ref{sec:theory} and run calibration in the range with the lowest values of $tr(B_1 + B_2)$. This improvement additional  requires approximation of the convolution maximum in larger data ranges.   

\item It is influenced to find out the minimal $h$ sufficient for bootstrap usage. Small $h$  improves Delay but makes unable to keep high level of  Precision and Recall in online mode. 
\end{enumerate}

\subsection{Experiments with real data}

Here data from 1972-75 Dow Jones Returns  \citet{BayesOnlineWeb}  describes several major events with potential macroeconomic effects (the most significant among them are the Watergate affair and the OPEC oil embargo). 
Convolutions plot with its upper bounds on this dataset appeared to be a nice illustration of multiscale search importance: CP near $t=325$ is better 	
perceptible when window size is equal to 30 and CP near $t=600$ has more perceptible convolution when window size is equal to 70. Two plots presented below includes convolutions with Static and Fitted Patterns, where one could remark better separability of convolution peaks in fitted case.       
 

\begin{figure}[ht!]
     \begin{center}
        \subfigure{%
            \label{fig:Precision_delta_N}
            \includegraphics[width=0.45\textwidth]{images/watergate1.pdf}
        }%
        \subfigure{%
           \label{fig:Recall_delta_N}
           \includegraphics[width=0.45\textwidth]{images/watergate2.pdf}
        }
         
         
%
    \end{center}
    \caption{%
        Data: daily returns of the Dow Jones Industrial Average from July 3, 1972 to June 30, 1975.
        Left plot -- convolution with static triangle pattern; right plot -- convolution with fitted triangle pattern. The time axis is in business days, conv 30 (50, 70) corresponds to pattern with window size 30 (50, 70), bound 30 (50, 70) corresponds to convolution upper bound. Three reference CP are presented: the conviction of G. Gordon Liddy and James W. McCord, Jr. on January 30, 1973 ($t = 142$); the beginning of the OPEC embargo against the United States on October 19, 1973 ($t = 325$); the resignation of President Nixon on August 9, 1974 ($t = 548$).
     }%
   \label{fig:subfigures}
\end{figure}


\subsection{Sources}

Demo of the LRTOnline method is available by link  \\ https://localcpdetector.shinyapps.io/localCP

Scala project with LRTOffline and LRTOnline methods could be cloned from \\
https://github.com/nazarblch/cpd \\
which also includes testing system for abrupt change points detection applications and generated data used in the experiments. 








\section{Conclusion}
This paper presented new change point detection method that works in offline  and online modes. The method accounts properties of LRT statistic, which has shifted $\chi^2$-distribution. Bootstrap technique appeared to be rather effective for  LRT  critical values calibration. Experiments and quality measurements confirm stability of the proposed algorithm in possibility to detect change points with different significance. The introduced concept of patterns allows to reveal different types of change point and filter regions with outliers.          



\bibliography{references}{}
\bibliographystyle{plain}

\end{document}
